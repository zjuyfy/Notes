\documentclass{article}
\usepackage{ctex}
\setCJKfamilyfont{hwxk}{STXingkai} 
\newcommand{\huawenxingkai}{\CJKfamily{hwxk}}
\usepackage{amsmath}
\usepackage{amssymb}
\usepackage{amsthm}
\usepackage{mathrsfs}
\usepackage{graphicx}
\newcommand{\xkuo}[1]{\left(#1\right)}
\newcommand{\dkuo}[1]{\left\lbrace#1\right\rbrace}
\newcommand{\zkuo}[1]{\left[#1\right]}
\newcommand{\piandao}[2][]{\frac{\partial #1}{\partial #2}}
\newcommand{\dao}[2][]{\frac{d#1}{d#2}}

\begin{document}

	\huawenxingkai\small
	\section{引力场}
	\subsection{能量动量张量}
	对于从$ x^i $到$ x'^i=x^i+\xi^i $的变换,有:$ g'^{ik}=g^{ik}+\delta g^{ik},\delta g^{ik}=\xi^{i;k}+\xi^{k;i} $
	\subsection{爱因斯坦场方程}
	换行\\
	引力场作用量:$ \delta S_g=C\cdot\delta \int R\sqrt{-g}d\Omega=C\cdot\int\xkuo{R_{ik}-\frac{1}{2}g_{ik}R}\delta g^{ik}\sqrt{-g}d\Omega$,$ C=-\frac{c^3}{16 \pi k} $\\
	与从$ S_g=C\cdot\int G\sqrt{-g}d\Omega $出发的变分(和对能量动量张量的变分差不多)比较,可得$$ R_{ik}-\frac{1}{2}g_{ik}R=\frac{1}{\sqrt{-g}}\dkuo{\piandao[\xkuo{G\sqrt{-g}}]{g^{ik}}-\piandao{x^l}\piandao[\xkuo{G\sqrt{-g}}]{\piandao[g^{ik}]{x^l}}} $$
	最小作用量:$\delta \xkuo{S_m+S_g}=0\Rightarrow-\frac{c^3}{16 \pi k}\int\xkuo{R_{ik}-\frac{1}{2}g_{ik}R-\frac{8\pi k}{c^4}T_{ik}}\delta g^{ik}\sqrt{-g}d\Omega=0 $\\
	由于$ \delta g^{ik} $的任意性,由上式得到引力场方程,即爱因斯坦场方程:$$ R_{ik}-\frac{1}{2}g_{ik}R=\frac{8\pi k}{c^4}T_{ik} \text{,或用混合指标表示:}R^i_{k}-\frac{1}{2}\delta^i_kR=\frac{8\pi k}{c^4}T^i_k $$
	或者由此式对两个指标缩并得到$ R=-\frac{8\pi k}{c^4}T $代入得$ R_{ik}=\frac{8\pi k}{c^4}\xkuo{T_{ik}-\frac{1}{2}g_{ik}T} $
	\subsubsection{性质}
	真空中,$ T_{ik}=0 $,所以有$ R_{ik}=0 $,但$ R_{iklm} $不一定为0,所以空间不一定平直\\
	由于对电磁场有$ T^i_i=0 $所以代入可知在仅有电磁场时R=0.
\end{document}