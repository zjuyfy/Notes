\documentclass[12pt, a4paper, oneside]{ctexbook}
        \usepackage{amsmath, amsthm, amssymb, amsfonts, bm, graphicx, hyperref, mathrsfs}
        \usepackage{tcolorbox}
        \usepackage{tikz, xcolor, environ, xparse, zhnumber}
        \usepackage{mathpazo}

        
        %设置页眉
        \usepackage{fancyhdr}
        \renewcommand{\headrulewidth}{1pt}
        \makeatletter
        \def\headrule{{\if@fancyplain\let\headrulewidth\plainheadrulewidth\fi
        \hrule\@height 1.0pt \@width\headwidth\vskip1pt%上面线为1pt粗  
        \hrule\@height 0.5pt\@width\headwidth  %下面0.5pt粗            
        \vskip-2\headrulewidth\vskip-1pt}      %两条线的距离1pt        
         \vspace{6mm}} 
        \pagestyle{fancy}
        
        
        
        
        
        
        \usetikzlibrary{shapes, decorations}
        
        %定义颜色
        \definecolor{lawcol}{RGB}{180,100,70}%推论环境的主色
        \definecolor{theocol}{RGB}{40,150,30}%定理环境的主色
        \definecolor{claimcol}{RGB}{150,170,20}
        \definecolor{thrmcol}{RGB}{18,29,80}%默认定理等环境的背景色
        \definecolor{thrmedge}{RGB}{12,133,211}%默认定理等环境的边界颜色
        \definecolor{hyperlinkcol}{RGB}{32,112,102}%链接颜色
        \definecolor{hyperfilecol}{RGB}{135,206,235}%文件颜色
        \definecolor{hyperurlcol}{RGB}{3,168,158}%网址颜色
        \definecolor{hypercitecol}{RGB}{150,140,130}%引用颜色
        \definecolor{hlback}{RGB}{207,255,207}%高亮颜色
        \definecolor{opcol}{RGB}{235,125,75}%op颜色
        \definecolor{facecol}{RGB}{122,180,245}%封面颜色
        \definecolor{pscol}{RGB}{44,80,99}%图案颜色

        
        %定义hyper的颜色
        \hypersetup{
          colorlinks=true,
          linkcolor=hyperlinkcol,
          filecolor=hyperfilecol,
          urlcolor=hyperurlcol,
          citecolor=hypercitecol,
        }
        
        %定义字体
        \setCJKfamilyfont{hwxk}{华文行楷}
        \newcommand{\huawenxingkai}{\CJKfamily{hwxk}}
        \setCJKfamilyfont{hwkt}{华文楷体}
        \newcommand{\huawenkaiti}{\CJKfamily{hwkt}}
        \setCJKfamilyfont{hwhp}{华文琥珀}
        \newcommand{\huawenhupo}{\CJKfamily{hwhp}}
        \setCJKfamilyfont{hwls}{华文隶书}
        \newcommand{\huawenlishu}{\CJKfamily{hwls}}
        \setmainfont{华文楷体}
        
        %定义高亮
        \newtcbox{\hlbox}[1][red]{on line, arc = 2pt, outer arc = 0pt,
          colback = hlback, colframe = #1!50!black,
          boxsep = 0pt, left = 1pt, right = 1pt, top = 2pt, bottom = 2pt,
          boxrule = 0pt, bottomrule = 1pt, toprule = 1pt}
        \newcommand{\hl}[1]{\hlbox{#1}}
        \newcommand{\optxt}[1]{\textcolor{opcol}{#1}}
        
        %定义小标题
        \newcommand{\tit}[1]{\begin{center}
          \large\hl{#1}
        \end{center}}
        
        
        
        
        %定义公式环境
        \newcommand{\newfancytheoremstyle}[5]{%
          \tikzset{#1/.style={draw=#3, fill=#2,very thick,rectangle,
              rounded corners, inner sep=10pt, inner ysep=20pt}}
          \tikzset{#1title/.style={fill=#3, text=#2}}
          \expandafter\def\csname #1headstyle\endcsname{#4}
          \expandafter\def\csname #1bodystyle\endcsname{#5}
        }
        
        \newfancytheoremstyle{fancythrm}{thrmcol!5}{thrmedge}{\bfseries\huawenhupo}{\huawenxingkai}
        
        \makeatletter
        \DeclareDocumentCommand{\newfancytheorem}{ O{\@empty} m m m O{fancythrm} }{
          %% 定义计数器
          \ifx#1\@empty
            \newcounter{#2}
          \else
            \newcounter{#2}[#1]
            \numberwithin{#2}{#1}
          \fi
          %% 定义 "newthem" 环境
          \NewEnviron{#2}[1][{}]{%
            \noindent\centering
            \begin{tikzpicture}
              \node[#5] (box){
                \begin{minipage}{0.93\columnwidth}
                  \csname #5bodystyle\endcsname \BODY~##1
                \end{minipage}};
              \node[#5title, right=10pt] at (box.north west){
                {\csname #5headstyle\endcsname #3 \stepcounter{#2}\csname the#2\endcsname\; ##1}};
              \node[#5title, rounded corners] at (box.east) {#4};
            \end{tikzpicture}
          }[\par\vspace{.5\baselineskip}]
        }
        
        
        \makeatother
        
         % 定义各个环境的的样式
         % \newfancytheoremstyle{<name>}{inner color}{outer color}{head style}{body style}
        \newfancytheoremstyle{fancytheo}{theocol!5}{theocol}{\huawenhupo}{\huawenxingkai}
        \newfancytheoremstyle{fancylaw}{lawcol!5}{lawcol}{\huawenhupo}{\huawenxingkai}
        \newfancytheoremstyle{fancyclaim}{claimcol!5}{claimcol}{\huawenhupo}{\huawenxingkai}
        
         % 定义各个新环境
         % \newfancytheorem[<number within>]{<name>}{<head>}{<symbol>}[<style>]
        \newfancytheorem[chapter]{define}{定义}{$\clubsuit$}
        \newfancytheorem[section]{deduce}{推论}{$\heartsuit$}[fancytheo]
        \newfancytheorem[section]{theorem}{定理}{$\spadesuit$}[fancytheo]
        \newfancytheorem[section]{attribute}{性质}{}[fancytheo]
        \newfancytheorem[chapter]{law}{定律}{$\clubsuit$}[fancylaw]
        \newfancytheorem[chapter]{claim}{声明}{\(\spadesuit\) }[fancyclaim]
        \newfancytheorem[chapter]{concept}{概念}{\(\spadesuit\)}[fancyclaim]
        
        \title{{\Huge{热力学笔记}}}
        \author{wave}
        \date{\today}
        \linespread{1.5}
        
        %设置章节标题样式\usepackage[english]{babel}
        \usepackage{blindtext}
        
        \usepackage[sc,compact,explicit]{titlesec} % Titlesec for configuring the header
        
        
        \usepackage{auto-pst-pdf} % Vectorian 装饰图案的 XeTeX 辅助 (见: https://tex.stackexchange.com/questions/253477/how-to-use-psvectorian-with-pdflatex)
        \usepackage{psvectorian} % Vectorian 中的装饰图案
        
        \let\clipbox\relax % PSTricks 已经定义了 \clipbox, 所以要去掉
        \usepackage{adjustbox} % 调整图案大小的
        
        \newcommand{\otherfancydraw}{% 定义图案
        \begin{adjustbox}{max height=0.5\baselineskip}% 根据行距设定高度,自己定
          \raisebox{-0.25\baselineskip}{
          \rotatebox[origin=c]{0}{% 旋转,自己定
            \psvectorian{84}% 图案,编号见 (http://melusine.eu.org/syracuse/pstricks/vectorian/psvectorian.pdf)
          }}%
        \end{adjustbox}%
        }
        
        % 画一条中间为图案的线 (见: https://tex.stackexchange.com/questions/15119/draw-horizontal-line-left-and-right-of-some-text-a-single-line/15122#15122)
        \newcommand*\ruleline[1]{\par\noindent\raisebox{.8ex}{\makebox[\linewidth]{\hrulefill\hspace{1ex}\raisebox{-.8ex}{#1}\hspace{1ex}\hrulefill}}}
        
        \titleformat% Formatting the header
          {\chapter} % command
          [block] % shape - Only managed to get it working with block
          {\normalfont\huawenlishu\huge} % format - Change here as needed
          {\centering 第\zhnum{chapter}章\\ \vspace{-0.6em}} % The Chapter N° label
          {0pt} % sep
          {\centering \ruleline{\otherfancydraw}\\ \vspace{-0.6em} % The horizontal rule
          \centering #1} % And the actual title
        
          \titleformat{\section}[block]{\huawenlishu\Large}{\thesection}{0pt}{\centering #1}
        
          %更改autoref的形式
          \def\equationautorefname{式}
          \def\footnoteautorefname{脚注}%
          \def\itemautorefname{项}%
          \def\figureautorefname{图}%
          \def\tableautorefname{表}%
          \def\partautorefname{篇}%
          \def\appendixautorefname{附录}%
          \def\chapterautorefname{章}%
          \def\sectionautorefname{节}%
          \def\subsectionautorefname{小小节}%
          \def\subsubsectionautorefname{subsubsection}%
          \def\paragraphautorefname{段落}%
          \def\subparagraphautorefname{子段落}%
          \def\FancyVerbLineautorefname{行}%
          \def\theoremautorefname{定理}%
          





          \newcommand{\xkuo}[1]{\left(#1\right)}
          \newcommand{\dkuo}[1]{\left\lbrace#1\right\rbrace}
          \newcommand{\akuo}[1]{\left[#1\right]}
          \newcommand{\jkuo}[1]{\left\langle#1\right\rangle}
          \newcommand{\daa}[1]{\par\textcolor{blue}{\huawenxingkai #1}}
          \newcommand{\wen}[1]{\mbox{#1}}
          \newcommand{\you}{\mbox{又}}
          \newcommand{\dang}{\mbox{当}}
          \newcommand{\yyou}{\mbox{有}}
          \newcommand{\qie}{\mbox{且}}
          \newcommand{\jishu}[2]{\sum #1_n(#2)}
          \newcommand{\jjishu}[2][x]{\sum |#2_n(#1)|}
          \newcommand{\xti}[3]{\[(#1)#2\]\da{#3}}
          \newcommand{\pian}[2]{\frac{\partial #1}{\partial #2}}
          \newcommand{\ppian}[2]{\frac{\partial^2 #1}{\partial #2^2}}
          \newcommand{\dao}[2]{\frac{d#1}{d#2}}
          \newcommand{\ddao}[2]{\frac{d^2#1}{d#2^2}}
          \newcommand{\cen}{^\circ C}
          \newcommand{\fah}{^\circ F}
          \newcommand{\ji}[2]{\int_{#1}^{#2}}
          \newcommand{\qh}[1]{\sum\limits_{#1}}
          \newcommand{\jji}[1]{\iint\limits_{#1}}
          \newcommand{\ppi}{\frac\pi2}
          \newcommand{\ege}{\frac{\sqrt2}{2}}
          \newcommand{\e}[1]{\times10^{#1}}
          \newcommand{\ti}[1]{\textbf{#1}}


          \newcommand{\ldotfill}[2]{\leavevmode\xleaders\hbox{\rule{2pt}{0.4pt}\ }\hfill\null}

          %作业用
          \newcounter{que}
          \setcounter{que}{1}
          \newenvironment{question}[1][\theque]{\vspace*{2cm}\par\noindent\hrule\vspace*{2pt}\hrule\vspace*{10pt}\noindent\bfseries\large Ex#1.}{\stepcounter{que}}
          \definecolor{anscolor}{RGB}{50,120,170}
          \newenvironment{answer}[1][答]{\par\centerline{\makebox[10cm]{\dotfill}}\par\hangafter1\hangindent2em\noindent\textbf{#1.}\huawenxingkai\color{anscolor}\\}{\par}

          %数学分析
          \newcommand{\ya}[4]{\frac{\partial(#1,#2)}{\partial(#3,#4)}}
          \newcommand{\zkya}[4]{\pd #1#3#4\pd #2#4#3-\pd #1#4#3\pd #2#3#4}

          %复变函数
          \newcommand{\wqji}{\int_{-\infty}^{\infty}}

          %热学
          \newcommand{\pd}[3]{\xkuo{\frac{\partial#1}{\partial#2}}_#3}

          %原子物理
          \newcommand{\bra}[1]{\left\langle #1 \right|}
          \newcommand{\ket}[1]{\left| #1 \right\rangle}
          \newcommand{\p}[1]{\partial_{#1}}%对下标的偏导
          \newcommand{\ep}[1]{\epsilon_{#1}}%全反对称张量
          \newcommand{\dt}[1]{\delta_{#1}}%delta张量

          %理论力学
          \newcommand{\keq}[2]{\pian{\mathscr{#1}}{#2}}
          \newcommand{\zkps}[3]{\pian{#1}{q_#3}\pian{#2}{p_#3}-\pian{#1}{p_#3}\pian{#2}{q_#3}}

          %正文
          \newcommand{\sub}[1]{\(_{#1}\)}
          \newcommand{\sps}[1]{\(^{#1}\)}



        \begin{document}
          \renewcommand*{\psvectorianDefaultColor}{pscol}%设定图案颜色
      
          \maketitle
      
          \pagenumbering{roman}
          \setcounter{page}{1}
      
          \begin{center}
              \Huge\huawenlishu{前言}
          \end{center}~\
      
          基础概念部分默认已经学过,就不加赘述,只给出简易的定义概念。
          ~\\
          \begin{flushright}
              \begin{tabular}{c}
                  何逸阳 \\
                  \today
              \end{tabular}
          \end{flushright}
          \begin{center}
              \Huge\huawenlishu{符号说明}
          \end{center}~\
      
      
          \newpage
          \pagenumbering{alph}
          \setcounter{page}{1}
          \tableofcontents
          \newpage
          \setcounter{page}{1}
          \pagenumbering{arabic}

          \part{热力学}
      
          \chapter{热力学定律}
      
          \section{基础概念}
          \begin{concept}
            \hl{热力学系统}指代任何宏观系统。
          \end{concept}
          \begin{concept}
            \hl{热力学参数}指可测量的热力学系统宏观参数。\optxt{它们由实验定义}。
          \end{concept}
          \begin{concept}
            \hl{热力学状态}由一组描述系统所需所有热力学参数的值的集合。\optxt{若不加说明,热力学状态指热力学平衡下的状态}
          \end{concept}
          \begin{concept}
            \hl{热力学平衡}指热力学状态不随时间变化。
          \end{concept}
          \begin{concept}
            \hl{状态方程}指描述热力学平衡下热力学系统的一组热力学参数之间的关系的方程。
          \end{concept}
          \begin{concept}
            \hl{热力学变化}指热力学状态的改变。
          \end{concept}
          \begin{concept}
            \hl{准静态}的热力学变化指极其缓慢以至于每个时刻可以近似看做热力学平衡的热力学变化。
          \end{concept}
          \begin{concept}
            当外部条件在时间上回溯它的历史时,如果热力学变化在时间上回溯它的历史,那么它就是\hl{可逆的}。
          \end{concept}
          \begin{concept}
            系统的\hl{P-V图}是状态方程的表面在P-V平面上的投影。特定类型的可逆变换会产生具有特定名称的路径,如等温线、绝热线等。
          \end{concept}
          \begin{concept}
            \hl{热}是均匀系统在不做功的情况下温度升高所吸收的热量。
          \end{concept}
          \begin{concept}
            \hl{热源}是一个大到任何有限热量的获得或损失都不会改变其温度的系统。
          \end{concept}
          \begin{concept}
            如果一个系统与外界之间不发生热交换,那么这个系统就是\hl{绝热}的。
          \end{concept}
          \begin{concept}
            如果一个热力学量与所考虑系统中的物质量成正比,则称为\hl{广度量};如果它与所考虑系统中的物质量无关,则称为\hl{强度量}。\optxt{对于一个很好的近似,热力学量要么是广度量,要么是强度量。}
          \end{concept}
          \begin{concept}
            \hl{理想气体}的参数是压强P、体积V、温度T和分子数n。状态方程由\hl{波义耳定律}给出:
            \begin{equation}
              \frac{P V}{N}= const.\quad \text{当温度恒定时}
            \end{equation}
          \end{concept}
          \begin{concept}
            由理想气体的状态方程可以定义一个温标——\hl{理想气体温度T}:
            \begin{equation}
              PV=NkT
            \end{equation}
          \end{concept}

          \section{热力学第一定律}
          热力学第零定律允许我们去定义温标这个状态函数,而热力学第一定律则允许我们去定义内能这个状态函数。

          \begin{law}
            \tit{热力学第一定律}
            在任意系统中,设$\Delta Q$代表系统吸收的净热量,\(\Delta W\)代表系统做的净功,则 \(\Delta U\):
            \begin{equation}
              \Delta U\triangleq \Delta Q-\Delta W
            \end{equation}
            对于给定了初始状态和最终状态的任何热力学变化都是相同的。
          \end{law}
          \begin{define}
            由热力学第一定律,我们可以定义给定热力学状态的\hl{内能}\(U\)(\optxt{由一个参照的初始状态沿任意方法变换得到给定状态}):
            \begin{equation*}
              U=U_0+\Delta Q-\Delta W
            \end{equation*}
            其\optxt{微分形式}为:
            \begin{equation}
              dU=dQ-dW
            \end{equation}
          \end{define}
          \begin{deduce}
            \label{ded:dQ}
            为求得热容的表达式,我们先写出\(dQ\)在各个参数下的表达式:
            \begin{align}
              \label{eq:dQPV}&d Q=\left(\frac{\partial U}{\partial P}\right)_V d P+\left[\left(\frac{\partial U}{\partial V}\right)_P+P\right] d V \\
              \label{eq:dQPT}&d Q=\left[\left(\frac{\partial U}{\partial T}\right)_P+P\left(\frac{\partial V}{\partial T}\right)_P\right] d T+\left[\left(\frac{\partial U}{\partial P}\right)_T+P\left(\frac{\partial V}{\partial P}\right)_T\right] d P \\
              \label{eq:dQVT}&d Q=\left(\frac{\partial U}{\partial T}\right)_V d T+\left[\left(\frac{\partial U}{\partial V}\right)_T+P\right] d V
              \end{align}
              由\autoref{eq:dQPT},可以得到定容热容和等压热容:
              \begin{equation}
                C_V\equiv \left(\frac{\Delta Q}{\Delta T}\right)_V=\left(\frac{\partial U}{\partial T}\right)_V
              \end{equation}
              \begin{equation}
                C_P\equiv \left(\frac{\Delta Q}{\Delta T}\right)_P=\left(\frac{\partial (U+PV)}{\partial T}\right)
              \end{equation}
          \end{deduce}
          \begin{define}
            由推导出的\hyperref[ded:dQ]{等压热容公式},我们定义\hl{焓}H:
            \begin{equation}
              H=U+PV
            \end{equation}
            于是有
            \begin{equation}
              C_P=\left(\frac{\partial H}{\partial T}\right)_P
            \end{equation}
          \end{define}

          \section{热力学第二定律}
          \begin{law}
            \hl{热力学第二定律的开尔文表述}不存在一个能自发地把吸收的热量完全转化为功的热机
          \end{law}
          \begin{law}
            \hl{热力学第二定律的克劳修斯表述}不存在一个能自发从低温热源把热量传给高温热源的热机
          \end{law}
          \begin{theorem}
            热力学第二定律的开尔文表述和克劳修斯表述互为充要条件。
          \end{theorem}
          \begin{define}
            \hl{卡诺热机}卡诺热机由四个不走组成:ab等温膨胀吸热,bc绝热膨胀,cd等温压缩,da绝热压缩。
          \end{define}
          \begin{define}
            热机的\hl{效率}定义为:
            \begin{equation}
              \eta=\frac{W}{Q_2}=1-\frac{Q_1}{Q_2}
            \end{equation}
          \end{define}
          \begin{theorem}
            \hl{卡诺定理}:给定两个温度之间,不存在比卡诺热机效率更高的热机。
          \end{theorem}
          \begin{deduce}
            两个给定温度下,所有卡诺热机拥有相同效率
          \end{deduce}
          \begin{define}
            由上一个推论,可以得到\hl{绝对温标}:\\
            取任意两个固定温度,卡诺热机的效率为\(\eta\),因此定义温标\(\theta_1,\theta_2\)满足:
            \begin{equation}
              \frac{\theta_1}{\theta_2}=1-\eta
            \end{equation}
            由于\(1-\eta\)总大于零,因此温度\(\theta\)总大于零。由于\(\frac{Q_1}{Q^2}=\frac{\theta_{n+1}}{\theta_n}\),因此我们假设有一组卡诺热机,每个对外做功\(W\),且上一个卡诺热机完全吸收下一个卡诺热机输出的热量\(Q^{n+1}_1=Q^n_2\triangleq Q_n\),于是有:
            \begin{equation}
              Q_{n+1}-Q_n=W,\quad \frac{Q_{n+1}}{Q_n}=\frac{\theta_{n+1}}{\theta_n}
            \end{equation}
            借此可以定义绝对温标。
          \end{define}
          
          \section{熵}
          与热力学第零定律和热力学第一定律一样,热力学第二定律允许我们定义一个新的状态函数——熵。

          \begin{theorem}\label{thrm:Clausius}
            \hl{克劳修斯定理}:
            对任意循环的热力学变化有
            \begin{equation}
              \oint \frac{dQ}{T}\leqslant 0
            \end{equation}
            当变化可逆时(即在相空间表现为一个闭合曲线),取等号。
          \end{theorem}
          \begin{define}\label{def:entropy}
            \hyperref[thrm:Clausius]{克劳修斯定理}告诉我们可以定义一个状态量\hl{熵}:
            \begin{equation}
              S(A)\equiv \int_O^A\frac{dQ}{T}
            \end{equation}
            其中\(O\)为原点状态,而\(A\)为相空间中任意状态。\\
            有定义显然有\(dS=\frac{dQ}{T}\)
          \end{define}
          \begin{deduce}\label{ded:deduce141}
            由\hyperref[thrm:Clausius]{克劳修斯定理}和\hyperref[def:entropy]{熵}的定义可知,对任意热力学变化有
            \begin{equation}
              \int_A^B\frac{dQ}{T}\leqslant S(B)-S(A)
            \end{equation}
          \end{deduce}
          \begin{deduce}
            由\hyperref[ded:deduce141]{上一个推论}可知,对任意孤立热力学系统(即dQ=0),熵永远增加:
            \begin{equation}
              S(B)-S(A)\geqslant 0
            \end{equation}
          \end{deduce}
          \begin{deduce}
            由\(dQ\)的表达式\autoref{eq:dQVT}可得
            \begin{equation}\label{eq:dSVT}
              dS=\frac{dQ}{T}=\xkuo{\frac{C_V}{T}}dT+\frac1T\akuo{\pd UVT+P}dV
            \end{equation}
            由于dS是全微分,可以得到
            \begin{equation}
              \left(\frac{\partial U}{\partial V}\right)_T=T\left(\frac{\partial P}{\partial T}\right)_V-P
            \end{equation}
            代回\autoref{eq:dSVT}得到
            \begin{equation}
              T d S=C_V d T+T\left(\frac{\partial P}{\partial T}\right)_V d V
            \end{equation}
            同样的方法利用\(dQ\)的另一个表达式\autoref{eq:dQPT}可得
            \begin{equation}
              T d S=C_P d T-T\left(\frac{\partial V}{\partial T}\right)_P d P
            \end{equation}
          \end{deduce}

          \section{热力学势能}
          我们引入两个辅助态函数,亥姆霍兹自由能A和吉布斯热力学势G(或吉布斯自由能),它们用来确定非孤立系统的平衡态。它们的定义如下:
          \begin{define}
            \hl{亥姆霍兹自由能}
            \begin{equation}
              A=U-TS
            \end{equation}
            其微分形式为
            \begin{equation}
              dA=-PdV-SdT
            \end{equation}
          \end{define}
          \begin{theorem}
            等温变换中,亥姆霍兹自由能A的变化等于系统所做的最大可能功的负值。
            \begin{equation}
              W\leqslant -\Delta A
            \end{equation}
          \end{theorem}
          \begin{deduce}
            对于\hl{保持恒定温度的力学孤立系统},亥姆霍兹自由能永不增加。因此其平衡状态是亥姆霍兹自由能最小的状态。
          \end{deduce}
          \begin{define}
            \hl{吉布斯自由能}
            \begin{equation}
              G=A+PV
            \end{equation}
            其微分形式为
            \begin{equation}\label{eq:dGPT}
              dG=VdP-SdT
            \end{equation}
          \end{define}
          \begin{theorem}
            在\hl{恒定的温度和压强}下,吉布斯势从不增加。因此其平衡状态是吉布斯自由能最小的状态
          \end{theorem}

          \section{热力学第三定律}

          \begin{law}
            \hl{热力学第三定律}:绝对零度时系统的熵是一个普适常数,可以取为零。
          \end{law}

          \chapter{热力学定律的应用}
          \section{相变的热力学描述}
          \begin{law}
            相变时,压强和温度不变
          \end{law}
          \begin{define}
            当一定质量的液体转化为气体时,尽管P和T保持不变,但系统的总体积会扩大。这样的转变被称为\hl{一阶转变}。
          \end{define}
          \begin{deduce}
            由于吉布斯势能必须最小:在改变占比\(\delta r_1=-\delta r_2=\delta r\)时,\(\delta G=0\)。设总质量为m,液态、气态单位质量吉布斯自由能(又被称为\hl{化学势})分别为\(g_1,g_2\),于是有
            \begin{equation}
              \delta G=\delta m\xkuo{r_1g_1+r_2g_2}=m\xkuo{g_1-g_2}\delta r=0
            \end{equation}
            于是得到一阶转变的平衡状态下有
            \begin{equation}
              g_1=g_2
            \end{equation}
            由吉布斯自由能的定义\autoref{eq:dGPT}得到
            
            \begin{equation}
              \left(\frac{\partial g}{\partial T}\right)_P=-s,\,\,\,\left(\frac{\partial g}{\partial P}\right)_T=v
            \end{equation}
            其中小写的\(v,s\)分别代表每单位质量的体积和熵。于是\(g_1\)的一阶导数与\(g_2\)在转变温度和压力下的一阶偏导是不同的(这就是为什么称为“一阶转变”)
          \end{deduce}
          \begin{deduce}
            \hl{克拉伯龙方程}:进一步推导可得\(\dao{P(T)}{T}=\frac{\Delta s}{\Delta v}\)

            定义\hl{转化潜热}\(l\):
            \begin{equation}
              l=T\Delta s
            \end{equation}
            于是克拉伯龙方程写为
            \begin{equation}
              \dao{P(T)}{T}=\frac{l}{T\Delta v}
            \end{equation}
          \end{deduce}
          \begin{define}
            \hl{n阶相变}:\\
            若\(\Delta s,\Delta v\)均为零(即两相密度一样,单位质量的熵也一样),则克拉伯龙方程失效。
            我们定义n阶相变:
            \begin{equation}
              \frac{\partial^n g_1}{\partial T^n} \neq \frac{\partial^n g_2}{\partial T^n} \quad \text { 并且 } \quad \frac{\partial^n g_1}{\partial P^n} \neq \frac{\partial^n g_2}{\partial P^n}
            \end{equation}
            同时其低阶偏导全部相同。

            现代一般只区分一阶和高阶转换,后者通常被不加区分地称为“二阶”转换。
          \end{define}
          \chapter{动力学}
          \begin{define}
            定义\(f(\vec r,\vec p,t)\)为t时刻下r,p范围dr,dp中的粒子数。则有
            \begin{equation}
              \int f(\vec r,\vec p,t)d\vec rd\vec p\equiv N
            \end{equation}
          \end{define}
          \begin{deduce}
            由动力学可知在外力\(\vec F\)下,不考虑碰撞时分布函数\(f\)满足
            \begin{equation}
              f(\vec r+\vec v\delta t,\vec p+\vec F\delta t,t+\delta t)=f(\vec r,\vec p,t)
            \end{equation}
          \end{deduce}
          \begin{deduce}
            考虑碰撞时,设
            \begin{equation}
              f(\vec{r}+\vec{v} \delta t, \vec{p}+\vec{F} \delta t, t+\delta t)=f(\vec{r}, \vec{p}, t)+\left(\frac{\partial f}{\partial t}\right)_{\mathrm{coll}} \delta t
            \end{equation}
            对\(\delta t\)展开后得到
            \begin{equation}
              \left(\frac{\partial}{\partial t}+\frac{\vec{p}}{m} \cdot \nabla_{\vec{r}}+\vec{F} \cdot \nabla_{\vec{p}}\right) f(\vec{r}, \vec{p}, t)=\left(\frac{\partial f}{\partial t}\right)_{\text {coll }}
            \end{equation}
          \end{deduce}
          \begin{deduce}
            \hl{玻耳兹曼输运方程}
            \begin{equation}
              \left(\frac{\partial}{\partial t}+\frac{\vec{p}_1}{m} \cdot \nabla_r+\vec{F} \cdot \nabla_{p_1}\right) f_1=\int d^3 p_2 d^3 p_1^{\prime} d^3 p_2^{\prime} \delta^4\left(P_{\mathrm{f}}-P_{\mathrm{i}}\right)\left|T_{\mathrm{fi}}\right|^2\left(f_2^{\prime} f_1^{\prime}-f_2 f_1\right)
            \end{equation}
          \end{deduce}
          \section{吉布斯系综}
          \begin{define}
            \hl{系综}指宏观条件完全相同的不同状态的系统的集合,由\(\Gamma\)空间中点的分布表示,一般是连续分布,可以由密度函数\(\rho(p,q,t)\)表示
          \end{define}
          \begin{theorem}
            \hl{刘维尔定理}:
            \begin{equation}
              \frac{\partial \rho}{\partial t}+\sum_{i=1}^{3 N}\left(\frac{\partial \rho}{\partial p_i} \dot{p}_i+\frac{\partial \rho}{\partial q_i} \dot{q}_i\right)=0
            \end{equation}
            也可以写成
            \begin{equation}
              \dao \rho t=0
            \end{equation}
            其几何意义是:如果我们关注\(\Gamma\)空间中一个代表点的运动,我们发现其邻域中的点的密度是恒定的。因此,代表性点的分布就像不可压缩的流体一样在\(\Gamma\)空间中移动。
          \end{theorem}
          \begin{define}
            系统的一个动态物理量\(O\)的\hl{观测值}定义为他在一个系综上的平均值:
            \begin{equation}
              \langle O\rangle=\frac{\int d^{3 N} p d^{3 N} q O(p, q) \rho(p, q, t)}{\int d^{3 N} p d^{3 N} q \rho(p, q, t)}
            \end{equation}
            称为物理量的\hl{系综平均值}
          \end{define}
          \section{BBGKY方程组}
          \begin{define}
            定义\hl{单粒子分布函数}
            \begin{align}
              f_{1}(\vec z, t) &\triangleq\left\langle\sum_{i=1}^N \delta^6\left(\vec{z}-\vec{z}_i\right) \right\rangle\\
              &=N \int d \vec z_2 \cdots d \vec z_N \rho(\vec z,\vec z_2, \ldots, \vec z_N, t)
            \end{align}
            其中,\(\vec z_i\triangleq(\vec p_i,\vec q_i)\),\(d\vec z_i\triangleq dp_xdp_ydp_zdxdydz\),\(\rho\)是概率密度函数\\
            因此,\(\int f_1 dz_1=1\)
          \end{define}
          \begin{define}
            \hl{s个粒子的分布函数}由以下公式导出:
            \begin{align}
              f_s(1, \ldots, z, t) \equiv \frac{N !}{(N-s) !} \int d z_{s+1} \cdots d z_N \rho(1, \ldots, N, t) \quad(s=1, \ldots, N)            \end{align}
          \end{define}
          \begin{define}
            定义\hl{\(h_s\)}:
            \begin{align}
              h_N(1, \ldots, N) & =\sum_{i=1}^N S_i+\frac{1}{2} \sum_{\substack{i, j=1 \\
              (i \neq j)}}^N P_{i j} \\
              S_i & \equiv \frac{\mathbf{p}_i}{m} \cdot \nabla_{r_i}+\mathbf{F}_i \cdot \nabla_{p_i} \\
              P_{i j} & \equiv \mathbf{K}_{i j} \cdot \nabla_{p_i}+\mathbf{K}_{j i} \cdot \nabla_{p_j}=\mathbf{K}_{i j} \cdot\left(\nabla_{p_i}-\nabla_{p_j}\right)
            \end{align}
            则有
            \begin{align}
              h_N& =\sum_{i=1}^s S_i+\sum_{s+1}^N S_i+\frac{1}{2} \sum_{\substack{i, j=1 \\
              (i \neq j)}}^s P_{i j}+\frac{1}{2} \sum_{\substack{i, j=s+1 \\
              (i \neq j)}}^N P_{i j}+\sum_{i=1}^s \sum_{j=s+1}^N P_{i j} \nonumber\\
              & =h_s(1, \ldots, s)+h_{N-s}(s+1, \ldots, N)+\sum_{i=1}^s \sum_{j=s+1}^N P_{i j}
            \end{align}
          \end{define}
          \begin{deduce}
            \hl{BBGKY递推公式}\(f_s(1,\ldots,s)\)
            \begin{equation}
              \begin{array}{r}
                \left(\frac{\partial}{\partial t}+h_s\right) f_s=-\sum_{i=1}^s \int d z_{s+1} \mathbf{K}_{i, s+1} \cdot \nabla_{p_t} f_{s+1} ,\quad
                (s=1, \ldots, N)
                \end{array}
            \end{equation}
          \end{deduce}
          \chapter{稀疏气体的平衡态}

          \part{统计物理}
          \chapter{经典统计力学}
          \begin{define}
            虽然我们有粒子相互作用的运动方程,但它们数量太多,自由度太多。符合相同方程,可以通过统计力学的方式来解决宏观的问题,以概率密度函数作为研究对象:
            \begin{equation}
              \rho(p_i,q_i,t)dp_1dq_1\ldots dp_ndq_n
            \end{equation}
            代表系统位于\(p_i,q_i\)位置的概率
          \end{define}
          \begin{define}
            \hl{系综}定义为
            平衡状态下,概率密度函数\begin{equation}
              \rho(p_i,q_i,t)\equiv\rho(p,q)
            \end{equation}
            
          \end{define}
          \begin{define}
            \hl{\(\Gamma\)空间}定义为由(p,q)张成的空间。\\
            系综当中的一个系统可以用\(\Gamma\)中的一个点来表示。
          \end{define}
          \begin{define}
            统计力学中,一个可测量物理量\(f(p,q)\)的\hl{系综平均值}定义为
            \begin{equation}
              \langle f\rangle\equiv\frac{\int f(p,q)\rho(p,q)dpdq}{\int\rho(p,q)dpdq}
            \end{equation}
          \end{define}
          \begin{claim}
            热力学极限下,系综平均值和最可能值相等。
          \end{claim}
          \section{微正则系综}
          \begin{define}
            \hl{微正则系综}的概率密度函数满足
            \begin{equation}
              \rho(p, q)= \begin{cases}\text { const. } & \text { if } E<H(p, q)<E+\Delta \\ 0 & \text { otherwise }\end{cases}
            \end{equation}
          \end{define}
          \begin{define}
            定义在\(\Gamma\)空间中的\hl{体积}为:
            \begin{equation}
              \Gamma(E)\equiv\int_{E<H(p,q)<E+\Delta}dpdq
            \end{equation}
          \end{define}
          \begin{define}
            \hl{熵}定义为
            \begin{equation}
              S(E,V)\triangleq k\ln\Gamma(E)
            \end{equation}
            其中\(k\)是玻耳兹曼常数。
          \end{define}
          \begin{attribute}
            熵是一个广延量:\(S=S_1+S_2\)
          \end{attribute}
          \begin{define}
            \hl{温度}定义为
            \begin{equation}
              \frac1T\triangleq \pian{S}{E}
            \end{equation}
          \end{define}
          \begin{define}
            \begin{equation}
              \Sigma(E)=\int_{H(p,q)<E}dpdq
            \end{equation}
          \end{define}
          \begin{define}
            \begin{equation}
              \omega(E)=\pian{\Sigma(E)}{E}
            \end{equation}
          \end{define}
          \begin{attribute}
            在近似下有:
            \begin{gather}
              \Gamma(E)=\omega(E)\Delta\\
              \Gamma(E)=\Sigma(E+\Delta)-\Sigma(E)\\
              S=k\ln\Gamma(E)\\
              S=k\ln\omega(E)\\
              S=k\ln\Sigma(E)
            \end{gather}
          \end{attribute}
          \begin{define}
            定义\hl{压强}\(P\):
            \begin{equation}
              P\triangleq T\xkuo{\pian SV}_E
            \end{equation}
          \end{define}
          \begin{attribute}
            \begin{equation}
              dE=TdS-PdV
            \end{equation}
          \end{attribute}

          \section{均分定理}
          \begin{claim}
            理想气体的粒子哈密顿量不考虑不同自由度之间的相互作用:
            \begin{equation}
              H=\frac1{2m}\sum p_i^2
            \end{equation}
            或对于谐振子
            \begin{equation}
              H=\frac1{2m}\sum p_i^2+\frac12\omega^2\sum q_i^2
            \end{equation}
          \end{claim}
          \begin{theorem}
            \hl{哈密顿量的均分定理}:设\(x_i\)等于\(p_i\)或\(q_i\),有
            \begin{equation}
              \jkuo{x_i\pian{H}{x_j}}=\delta_{ij}kT
            \end{equation}
            而对于形如\(H=\sum\limits_iA_ip_i^2+\sum\limits_iB_iq_i^2\)这样的哈密顿量,有
            \begin{equation}
              \jkuo{2H}=\jkuo{\sum_{i=1}^n p_i\pian{H}{p_i}+\sum_{i=1}^ma_i\pian{H}{q_i}}=(m+n)kT
            \end{equation}
          \end{theorem}
          \begin{theorem}
            \hl{理想气体能量均分定理}对自由度为\(f\)理想气体,
            \begin{equation}
              \jkuo{H}=\frac12fkT
            \end{equation}
          \end{theorem}
          \section{经典理想气体}
          \begin{claim}
            这个部分理想气体粒子的哈密顿量为\(H=\frac1{2m}\sum p_i^2\)
          \end{claim}
          \begin{deduce}
            \begin{equation}
              \Sigma(E)=V^N\Omega_{3N}(R)=C_{3N}\akuo{V(2mE)^{3/2}}^N
            \end{equation}
            其中,\(R=\sqrt[]{2mE}\),N为粒子数,\(\Omega_n(R)=\frac{2\pi^{n/2}}{\Gamma(n/2+1)}R^n\)为n维球体体积(动量小于\(R\)的球体)
          \end{deduce}
          \begin{deduce}
            理想气体的熵:
            \begin{equation}
              S(E,V)=k\ln\Sigma(E)\approx N k \ln \left[V\left(\frac{4 \pi m E}{3 N}\right)^{3 / 2}\right]+\frac{3}{2} N k
            \end{equation}
            于是求解得到
            \begin{equation}
              U(S, V) \equiv E=\frac{3 N}{4 \pi m V^{2 / 3}} \quad \exp \left(\frac{2 S}{3 N k}-1\right)
            \end{equation}
            以此得到剩下热力学量:
            \begin{align}
              T & =\left(\frac{\partial U}{\partial S}\right)_V=\frac{2 U}{3 N k}, \quad\text{即} U=\frac{3}{2} N k T \\
              C_V & =\frac{3}{2} N k \\
              P & =-\left(\frac{\partial U}{\partial V}\right)_S=\frac{N k T}{V}
              \end{align}
          \end{deduce}
          \begin{define}
            为了使两个相同的理想气体相互扩散时熵不增加,引入一个\hl{吉布斯因子}:
            \begin{equation}
              \Sigma(E)=\frac1{N!}\int_{H<E}d^Npd^Nq
            \end{equation}
            此因子与量子力学原理有关。
          \end{define}
          \section{麦克斯韦-玻耳兹曼分布}
          \begin{claim}
            此部分气体粒子能量相互独立,类似微正则分布。
          \end{claim}
          \begin{define}
            \hl{\(\mu\)空间}定义为单个粒子的自由度张成的相空间\(\xkuo{p,q}\)
          \end{define}
          \begin{theorem}
            \begin{center}
              \hl{玻耳兹曼统计理论}
            \end{center}
            将总粒子数分为\(K\)个部分之和
            \begin{equation}\label{eqn:N}
              N=\sum_{l=1}^Kn_l
            \end{equation}
            于是
            \begin{equation}\label{eqn:E}
              E=\sum_{l-1}^K \epsilon_ln_l
            \end{equation}
            则系统在\(\Gamma\)空间中占据的体积为
            \begin{equation}
              \Omega\xkuo{n_1,\ldots,n_K}\propto \frac{N!}{n_1!\ldots n_K!}\prod_{l=1}^Kg_l^{n_l}
            \end{equation}
            其中\(g_l\)是引入的乘子,最终需要使它等于1\\
            通过变分法可以求得最可能的\(n_l\)分布:
            \begin{equation}
              \bar n_l=g_le^{-\alpha-\beta\epsilon_l}
            \end{equation}
            其中\(\alpha\)、\(\beta\)是拉格朗日乘子,由\autoref{eqn:N}和\autoref{eqn:E}确定(即由总的粒子数和能量确定)。
          \end{theorem}
          \begin{define}
            定义\hl{配分函数}:
            \begin{equation}
              Z(\beta, y)=\sum_l e^{-\beta \epsilon_l}=\int_{\epsilon \leq E} d p d q e^{-\beta \epsilon(p, q, y)}
            \end{equation}
          \end{define}
          \begin{deduce}
            代入可得
            \begin{align}
                N & =e^{-\alpha} Z(\beta, y) \quad \text { 即 } \quad \alpha=\ln \frac{Z(\beta, y)}{N} \\
                E & =-N \frac{\partial \ln Z(\beta, y)}{\partial \beta}
            \end{align}
            由\autoref{eqn:N}可以得到系统能量变化有两部分组成:
            \begin{equation}
              dE=\sum \bar n_ld\epsilon_l+\sum e_ld\bar n_l
            \end{equation}
            先分析前一部分,设\(y_k\)是与外部环境相互作用的自由度,则
            \begin{equation}
              \begin{aligned}
                \sum_l \bar{n}_l \mathrm{~d} \epsilon_l & =\sum_{l, k} \bar{n}_l \frac{\partial \epsilon_l}{\partial y_k} \mathrm{~d} y_k \\
                & =\sum_k\left(\sum_l \bar{n}_l \frac{\partial \epsilon_l}{\partial y_k}\right) \mathrm{d} y_k\\
                &=-\sum_k Y_k \mathrm{d} y_k
                \end{aligned}
            \end{equation}
            其中\(Y_k\)是作用力
            \begin{equation}
              Y_k \equiv-\sum_l \bar{n}_l \frac{\partial \epsilon_l}{\partial y_k}=\frac{N}{\beta} \frac{\partial \ln Z(\beta, y)}{\partial y_k}
            \end{equation}
            因此,\hl{压力}
            \begin{equation}
              P=\frac{N}{\beta} \frac{\partial \ln Z(\beta, V)}{\partial V}
            \end{equation}
            因此另一部分是\hl{热量}
            \begin{equation}
              dQ=\sum_l \epsilon_ld\bar n_l
            \end{equation}
            计算得
            \begin{equation}
              \mathrm{d} Q=\frac{N}{\beta} \mathrm{d}\left(\ln Z(\beta, y)-\beta \frac{\partial \ln Z(\beta, y)}{\partial \beta}\right)
            \end{equation}
          \end{deduce}
          \begin{deduce}
            因此可以得到\hl{温度}:
            \begin{equation}
              \beta=\frac{1}{kT}
            \end{equation}
            于是\hl{熵}
            \begin{equation}
              \begin{aligned}
                \mathrm{d} S & =N k \mathrm{~d}\left(\ln Z(\beta, y)-\beta \frac{\partial \ln Z(\beta, y)}{\partial \beta}\right) \\
                \text{即}S & =N k\left(\ln Z(\beta, y)-\beta \frac{\partial \ln Z(\beta, y)}{\partial \beta}\right)+\text { const }
                \end{aligned}
            \end{equation}
          \end{deduce}

          \chapter{正则系综和巨正则系综}
          \section{正则系综}
          \begin{deduce}
            设系统分为两部分,\(E_1<<E_2,N_1<<N_2\)\\
            那么系统1处于\((p_1,q_1)\)状态的概率为
            \begin{equation}
              \rho(p,q)\propto \int_{E_2=E-E_1}d_2dq_2=\Gamma_2(E-E_1)
            \end{equation}
            泰勒展开可以得到
            \begin{equation}
              \begin{aligned}
                k \ln \Gamma_2\left(E-E_1\right) & =k \ln \Gamma_2(E)-\left.E_1 \frac{\partial k \ln \Gamma_2\left(E_2\right)}{\partial E_2}\right|_{E_2=E}+\cdots \\
                & =S_2(E)-\frac{E_1}{T}
                \end{aligned}
            \end{equation}
            于是有 
            \begin{equation}
              \rho(p, q) \propto e^{-H(p, q) / k T}
            \end{equation}
          \end{deduce}
          \begin{define}
            \hl{正则系综的概率分布函数}定义为 
            \begin{equation}
              \rho(p, q) \propto e^{-H(p, q) / k T}
            \end{equation}
            其宏观条件为:
            \begin{enumerate}
              \item N固定
              \item 与恒温热源T保持平衡
            \end{enumerate}
          \end{define}
          \begin{define}
            定义正则系综的\hl{配分函数}:
            \begin{equation}
              Q_N(V, T) \equiv \frac{1}{N !} \int \mathrm{d}^{3 N} p \mathrm{~d}^{3 N} q e^{-H / k T}=e^{-A(V,T)/kT}
            \end{equation}
            其中,\(A\)是\hl{自由能}
          \end{define}
          \begin{theorem}
            物理量的\hl{观测值}:
            \begin{equation}
              \jkuo{O}=\frac{1}{Q_N N !} \int  e^{-\beta H} O\mathrm{d} p \mathrm{~d} q
            \end{equation}
          \end{theorem}
          \begin{theorem}
            \begin{equation}
              \begin{aligned}
                <H^2>-<H>^2 & =-\frac{\partial U}{\partial \beta}=k T^2 \frac{\partial U}{\partial T} \\
                & =k T^2 C_V
              \end{aligned}
            \end{equation}
          \end{theorem}
          \begin{deduce}
            \hl{玻耳兹曼统计理论}
            配分函数与自由能的关系可由\(Q_N(V,T)\)得到:
            \begin{equation}
              \frac{1}{N !}[Z(\beta, V)]^N=e^{-\beta A(V, T)}
            \end{equation}
            于是有
            \begin{equation}
              A(V, T)  =-N k T \ln \frac{Z(\beta, V)}{N}
            \end{equation}
            因此
            \begin{equation}
              \begin{aligned}
                S & =-\left(\frac{\partial A}{\partial T}\right)_V \\
                & =N k \ln \frac{Z(\beta, V)}{N}+\frac{N}{\beta} \frac{\partial \ln Z(\beta, V)}{\partial \beta} \frac{\partial \beta}{\partial T} \\
                & =N k\left(\ln \frac{Z(\beta, V)}{N}-\beta \frac{\partial \ln Z(\beta, V)}{\partial \beta}\right)
                \end{aligned}
            \end{equation}
          \end{deduce}
          \section{正则系综的应用}
          \begin{theorem}
            \hl{能均分定理}
            若
            $$
            H=\frac{1}{2} \sum_i a_i p_i^2+U\left(\left\{q_i\right\}\right)
            $$
            则
            $$
            <\frac{1}{2} a_i p_i^2>=\frac{1}{2} k T
            $$
          \end{theorem}
          \begin{theorem}
            \hl{理想气体}
            \begin{equation}
              H=\frac{1}{2 m} \sum_i p_i^2
            \end{equation}
            代入可得
            \begin{align}
                Q_N(V, T) & =\frac{1}{N !} \int \mathrm{d}^{3 N} p \mathrm{~d}^{3 N} q e^{-\beta \sum_i p_i^2 / 2 m} \\
                & =\frac{V^N}{N !}\left(\int \mathrm{d}^3 p e^{-\beta p^2 / 2 m}\right)^N \\
                & =\frac{V^N}{N !}[q(\beta)]^N \\
                q(\beta) & =\int_0^{\infty} \mathrm{d} p p^2 e^{-p^2 / 2 m k T} \int \mathrm{d} \Omega \\
                & =(\pi 2 m k T)^{3 / 2}\\
                \Rightarrow A(V, T) & =-k T \ln Q_N(V, T) \\
                  & =-N k T\left(\ln V / N+\frac{3}{2} \ln \pi 2 m k T+1\right)
            \end{align}
            进而可以导出\(S\)、\(P\)、\(U\)
          \end{theorem}

          \section{Ising模型}

          \section{巨正则系综}
          \begin{define}
            \begin{equation}
              \rho\left(p_1, q_1, N_1\right)=\frac{Q_{N_2}\left(V_2, T\right)}{Q_N(V, T)} \frac{e^{-\beta H\left(p_1, q_1, N_1\right)}}{N_{1} !}
            \end{equation}
            在\(N_2>>N_1,\,V_2>>V_1\)的近似下对\(A\)泰勒展开可以得到
            \begin{equation}
              \rho(p, q, N)=\frac{e^{-\beta P V}}{N !} e^{\beta \mu N-\beta H(p, q, N)}
            \end{equation}
          \end{define}
          \begin{define}
            定义配分函数
            \begin{equation}
              Z(\mu, V, T) \equiv \sum_{N=0}^{\infty} e^{\beta \mu N} Q_N(V, T)
            \end{equation}
          \end{define}
          \begin{deduce}
            \begin{equation}
              \beta P V=\ln Z(\mu, V, T)
            \end{equation}
            \begin{equation}
              \begin{aligned}
                \bar{N} & =\langle N\rangle=\frac{1}{Z(\mu, V, T)} \sum_{N=0}^{\infty} N e^{\beta \mu N} Q_N(V, T) \\
                & =k T \frac{\partial}{\partial \mu} \ln Z(\mu, V, T)
                \end{aligned}
            \end{equation}
          \end{deduce}
          \begin{deduce}
            \begin{equation}
              \mathrm{d} A=-P \mathrm{~d} V-S \mathrm{~d} T+\mu \mathrm{d} \bar{N}
            \end{equation}
          \end{deduce}
          \begin{deduce}
            \begin{equation}
              -\beta \mu=-\beta \frac{\partial A}{\partial N}=\ln (Z(\beta, V) / N)
            \end{equation}
          \end{deduce}
          \chapter{量子统计}
          \section{复习量子力学}
          \begin{define}
            \hl{希尔伯特空间} 
          \end{define}
          \begin{define}
            微观系统在\(t\)时刻的一个\hl{状态}对应于希尔伯特空间中的一个点:
            \begin{equation}
              \psi=\sum_n C_n(t) \phi_n
            \end{equation}
          \end{define}
          \begin{define}
            \hl{内积}
          \end{define}
          \begin{define}
            \hl{物理可观测值平均值}
          \end{define}
          \begin{define}
            \hl{薛定谔方程}
          \end{define}
          \begin{define}
            \hl{算符的矩阵表示}
          \end{define}
          \begin{define}
            \hl{狄拉克符号}
          \end{define}
          \begin{claim}
            自然界中有两种粒子:
            \begin{enumerate}
              \item \hl{玻色子}\\
              波函数在任意一对粒子坐标的交换下都是对称的
              \item \hl{费米子}\\
              波函数在任意一对粒子坐标的交换下是反对称的
            \end{enumerate}
            因此费米子有\hl{泡利不相容原理}
          \end{claim}
          \section{量子统计的假设}
          \begin{claim}
            设在\(\dkuo{\phi_n}\)表象下可观测量为
            \begin{equation}
              \overline{\langle O(x)\rangle}=\overline{(\psi, O(x) \psi)}=\sum_{n, m} \overline{C_n^*(t) C_m(t)}\left(\phi_n, \hat{O} \phi_m\right)
            \end{equation}
            微正则系综:
            $$\overline{C_n^*(t) C_m(t)}=0 \quad n \neq m$$
            $$\overline{C_n^*(t) C_n(t)}= \begin{cases}|b_n|^2 & E<E_n<E+\Delta \\ 0 & \text { otherwise }\end{cases}$$
          \end{claim}
          \begin{define}
            定义\hl{密度矩阵}:
            \begin{equation}
              \rho_{n m} \equiv\left(\phi_n, \rho \phi_m\right) \equiv \delta_{n m}\left|b_n\right|^2
            \end{equation}
            在狄拉克符号中,
            \begin{equation}
              \rho=\sum_n\left|\phi_n\right\rangle\left|b_n\right|^2\left\langle\phi_n\right| .
            \end{equation}
          \end{define}
          \begin{deduce}
            \begin{equation}
              \langle O\rangle=\frac{\sum_n\left\langle\phi_n|O \rho| \phi_n\right\rangle}{\sum_n\left\langle\phi_n|\rho| \phi_n\right\rangle}=\frac{T_r(O \rho)}{T_r \rho}
            \end{equation}
          \end{deduce}
          \section{系综}
          \begin{define}
            \hl{微正则系综}:\(|b_n|^2=1\)
            \begin{equation}
              \rho=\sum_{E<E_n<E+\Delta}\left|\phi_n\right\rangle\left\langle\phi_n\right|
            \end{equation}
            定义体积:
            \begin{equation}
              \Gamma(E)=T_r \rho
            \end{equation}
            即能量在\(E\)到\(E+\Delta\)之间的特征态数量
          \end{define}
          \begin{define}
            定义\hl{熵}:
            \begin{equation}
              S(E, V)=k \ln \Gamma(E)
            \end{equation}
          \end{define}
          \begin{define}
            \hl{正则系综}
            \begin{equation}
              \rho_{n m}=\delta_{n m} e^{-\beta E_n} \quad \beta=\frac{1}{k T}
            \end{equation}
            \begin{equation}
              \hat\rho=e^{-\beta \hat H}
            \end{equation}
          \end{define}
          \begin{define}
            \hl{配分函数}
            \begin{equation}
              \begin{aligned}
                Q_N(V, T) & =T_r \rho \\
                & =\sum_n e^{-\beta E_n}
                \end{aligned}
            \end{equation}
          \end{define}
          \begin{deduce}
            对于可观测量,
            \begin{equation}
              \langle O\rangle=\frac{T_r(O \rho)}{T_r \rho}=\frac{1}{Q_N} T_r\left(O e^{-\beta H}\right)
            \end{equation}
            在能量表象下,
            \begin{equation}
              \langle O\rangle=\frac{1}{Q_N} \sum_n O_{n n} e^{-\beta E_n}
            \end{equation}
          \end{deduce}
          \begin{define}
            \hl{巨正则配分函数}
            \begin{equation}
              Z(\mu, V, T)=\sum_{N=0}^{\infty} e^{\beta \mu N} \cdot Q_N(V, T)
            \end{equation}
            \begin{equation}
              \langle O\rangle=\frac{1}{Z(\mu, V, T)} \sum_{N=0}^{\infty} e^{\beta \mu N}\langle O\rangle_N
            \end{equation}
            \begin{equation}
              \rho=e^{-\beta(H-\mu N)}
            \end{equation}
          \end{define}
          \begin{deduce}
            由N个准独立的双能级子系统组成的系统,其内能
            \begin{equation}
              U=\langle E\rangle=\frac{N \epsilon \exp (-\beta \epsilon)}{1+\exp (-\beta \epsilon)}
            \end{equation}
            由
            \begin{equation}
              Q_N=Z^N,\,A=-k T \ln Q_N
            \end{equation}
            得到
            \begin{equation}
              \mu=\left(\frac{\partial A}{\partial N}\right)_T=-k T \ln [1+\exp (-\beta \epsilon)]
            \end{equation}
          \end{deduce}
          \section{理想气体的微正则系综}

          \begin{deduce}
            粒子能量\(\epsilon_p=\frac{p^2}{2 m}\),其中
            \begin{equation}
              \begin{gathered}
                \vec{p}=\frac{2 \pi \hbar}{L} \vec{m} \Leftrightarrow\left\{\begin{array}{l}
                p_x=2 \pi \hbar m_x / L \\
                p_y=2 \pi \hbar m_y / L \\
                p_z=2 \pi \hbar m_z / L
                \end{array}\right. \\
                m_x, m_y, m_z=0, \pm 1, \pm 2, \cdots
                \end{gathered}
            \end{equation}
          \end{deduce}
          \begin{deduce}
            对于\hl{玻色子},考虑将系统等效为\(i\)个具有\(g_i\)个简并度的能级,每个状态\(n_i\)个粒子。
            则状态数
            \begin{equation}
              W\left\{n_i\right\}=\prod_i \omega_i=\prod_i \frac{\left(n_i+g_i-1\right) !}{n_{i} !\left(g_i-1\right) !}
            \end{equation}
            \begin{equation}
              \omega_i=\frac{\left(n_i+g_i-1\right) !}{n_{i} !\left(g_i-1\right) !}
            \end{equation}

          \end{deduce}
          \begin{deduce}
            对于\hl{费米子}每个状态最多仅有一个粒子,即\(n_i=0,1\),于是
            \begin{equation}
              \omega_i=\frac{g_{i} !}{n_{i} !\left(g_i-n_i\right) !} \quad W\left\{n_i\right\}=\prod_i \omega_i
            \end{equation}
          \end{deduce}
          \begin{deduce}
            \begin{equation}
              \begin{aligned}
                S & =k \ln \Gamma(E) \\
                & =k \ln \sum_{\left\{n_i\right\}} W\left\{n_i\right\}\\
                &=k\ln W{\bar n_i}
                \end{aligned}
            \end{equation}
            其中,\(\bar n_i\)为占据主导地位的粒子数分布
          \end{deduce}
          \begin{deduce}
            通过变分法求得\(W{n_i}\)的最大值,再把\(g_1,\epsilon_i\)替换成\(1,\epsilon_{\vec p}\):
            \begin{enumerate}
              \item 玻色子
              \begin{equation}
                \bar{n}_{\vec{p}}=\frac{1}{e^{\alpha+\beta \epsilon_{\vec{p}}}-1}
              \end{equation}
              \item 费米子
              \begin{equation}
                \bar{n}_{\vec{p}}=\frac{1}{e^{\alpha+\beta \epsilon_{\vec{p}}}+1}  
              \end{equation}
            \end{enumerate}
            其中,\(\beta=1/kT,\quad\alpha=-\beta \mu\)
          \end{deduce}
          
          \section{理想玻耳兹曼气体}
          \begin{deduce}
            假设粒子是可区分的,并且不遵循泡利原理,同上的方法可导出
            \begin{equation}
              W\left\{n_i\right\}=\prod_i \frac{g_i^{n_i}}{n_{i} !}
            \end{equation}
            \begin{equation}
              \bar{n}_{\vec{p}}=e^{-\alpha-\beta \epsilon_{\vec{p}}}
            \end{equation}
          \end{deduce}
          \begin{deduce}
            \begin{equation}
              \begin{aligned}
                N&=e^{-\alpha} \sum_{\vec{p}} e^{-\beta \epsilon_{\vec{p}}}
                & =e^{-\alpha} \frac{V}{h^3} \int_{-\infty}^{+\infty} \mathrm{d}^3 p e^{-\beta p^2 / 2 m} \\
                & =e^{-\alpha} \frac{V}{h^3} \int \mathrm{d} \Omega \int_0^{+\infty} \mathrm{d} p \cdot p^2 e^{-\beta p^2 / 2 m}\\
                &=e^{-\alpha}\frac{V}{\lambda^3}
              \end{aligned}
            \end{equation}
            其中
            \begin{equation*}
              \lambda=\sqrt{\frac{2 \pi \hbar^2}{m k T}}
            \end{equation*}
          \end{deduce}
          \begin{deduce}
            用同样方法可以得到
            \begin{equation}
              E=\frac{3N}{2\beta}
            \end{equation}
            \begin{equation}
              S=\frac{3}{2}kN-N\ln\akuo{\frac NV\xkuo{\frac{2\pi\hbar^2}{mkT}}^{3/2}}
            \end{equation}
            这些结果与经典相同
          \end{deduce}

          \section{理想气体巨正则系综}
          \begin{deduce}
            由于
            \begin{equation}
              Z(\mu, V, T)= \begin{cases}\prod_{\vec{p}}\left(1-e^{\beta \mu-\beta \epsilon_{\vec{p}}}\right)^{-1} & \text { 玻色子 } \\ \prod_{\vec{p}} 1+e^{\beta \mu-\beta \epsilon_{\vec{p}}} & \text { 费米子 }\end{cases}
            \end{equation}
          \end{deduce}
          
          \begin{deduce}
            \begin{equation}
              \begin{aligned}
                \bar{N} & =\frac{1}{Z(\mu, V, T)} \sum_{N=0}^{\infty} \sum_{\substack{\left\{n_{\vec{p}}\right\}}} N e^{\beta \mu N} e^{-\beta \sum \epsilon_{\vec{p}} n_{\vec{p}}} \\
                & =\frac{1}{\beta} \frac{\partial}{\partial \mu} \log Z(\mu, V, T) 
                \end{aligned}
            \end{equation}
            \begin{equation}
              \begin{aligned}
                \bar{n}_{\vec{p}} & =\frac{1}{Z(\mu, V, T)} \sum_{N=0} e^{\beta \mu N} \sum_{\left\{n_{\vec{p}}\right\}} n_{\vec{p}} e^{-\beta \sum \epsilon_{\vec{p}} n_{\vec{p}}} \\
                & =-\frac{1}{\beta} \frac{\partial}{\partial \epsilon_{\vec{p}}=N} \log Z(\mu, V, T) \\
                & =\frac{1}{e^{-\beta \mu+\beta \epsilon_{\vec{p}}} \mp 1}
              \end{aligned}
            \end{equation}
          \end{deduce}
          \begin{deduce}
            对于费米子
            \begin{equation}
              \begin{aligned}
                \frac{P V}{k T} & =\sum_{\vec{p}} \log \left(1+e^{\beta \mu-\beta \epsilon_{\vec{p}}}\right) \\
                & =\frac{V}{h^3} \int \mathrm{d} \Omega \int \mathrm{d} p p^2 \log \left(1+e^{\beta \mu-\beta p^2 / 2 m}\right) \\
                \frac{P}{k T} & =\frac{4 \pi}{h^3} \int_0^{\infty} \mathrm{d} p p^2 \log \left(1+e^{\beta \mu-\beta p^2 / 2 m}\right) \\
                \frac{\bar{N}}{V} & =\frac{4 \pi}{h^3} \int_0^{\infty} \mathrm{d} p p^2\left(e^{-\beta \mu+\beta p^2 / 2 m}+1\right)^{-1}
                \end{aligned}
            \end{equation}
            对于玻色子,当\(\mu\rightarrow0\)时\(\ln(1-e^{\beta\mu})\)发散,因此分离\(\vec p=0\)的情况
            \begin{equation}
            \begin{aligned}
              & \frac{P}{k T}=-\frac{4 \pi}{h^3} \int_0^{\infty} \mathrm{d} p p^2 \log \left(1-e^{\beta \mu-\beta p^2 / 2 m}\right)-\frac{1}{V} \log \left(1-e^{\beta \mu}\right) \\
              & \qquad \frac{\bar{N}}{V}=\frac{4 \pi}{h^3} \int_0^{\infty} \mathrm{d} p p^2 \frac{1}{e^{-\beta \mu+\beta p^2 / 2 m}-1}+\frac{1}{V} \frac{e^{\beta \mu}}{1-e^{\beta \mu}}
              \end{aligned}
            \end{equation}
          \end{deduce}

          \chapter{量子统计力学的应用}

          \section{光子}

          \begin{claim}
            对光子而言,电场
            \begin{equation}
              \vec{E}(\vec{r}) =\vec{\epsilon} e^{i(\vec{k} \cdot \vec{r}-\omega t)} 
            \end{equation}
            因此有
            \begin{equation}
              \begin{aligned}
                \text { energy } & =\hbar \omega \\
                \text { momentum } & =\hbar \vec{k}, \quad \omega=c|\vec{k}| \\
                \text { polarization } & =\vec{\epsilon}, \quad|\vec{\epsilon}|=1, \vec{k} \cdot \vec{\epsilon}=0
                \end{aligned}
            \end{equation}
          \end{claim}
          \begin{deduce}
            
            \begin{equation}
              \begin{aligned}
                Z & =\prod_{\vec{k}, \vec{\epsilon}} \sum_{n=0}^{\infty} e^{-\beta \hbar \omega n}=\prod_{\vec{k}, \vec{\epsilon}} \frac{1}{1-e^{-\beta \hbar \omega}} \\
                \log Z & =-2 \sum_{\vec{k}} \log \left(1-e^{-\beta \hbar \omega}\right)
                \end{aligned}
            \end{equation}
            其中因子2来自\(\epsilon\)的和
            \begin{equation}
              \left\langle n_{\vec{k}}\right\rangle=-\frac{\partial}{\partial(\beta \hbar \omega)} \log Z=\frac{2}{e^{\beta \hbar \omega}-1}
            \end{equation}
            \begin{equation}
              U=\sum_{\vec k}\hbar\omega\jkuo{n_{\vec k}}
            \end{equation}
            \begin{equation}
              P=-\left(\frac{\partial A}{\partial V}\right)_T=k T\left(\frac{\partial \log Z(V, T)}{\partial V}\right)_T=\frac{1}{3V}\sum_{\vec k}\hbar\omega\jkuo{n_{\vec k}}
            \end{equation}
            因此 
            \begin{equation}
              PV=\frac13U
            \end{equation}
          \end{deduce}
          \begin{deduce}
            \begin{equation}
              \begin{aligned}
                & U=\frac{2 V}{(2 \pi)^3} \int_0^{\infty} \mathrm{d} k 4 \pi k^2 \frac{\hbar \omega}{e^{\beta \hbar \omega}-1} \\
                &=\frac{V \hbar}{\pi^2 c^3} \int_0^{\infty} \mathrm{d} \omega \frac{\omega^3}{e^{\beta \hbar \omega}-1} \\
                & \frac{U}{V}=\int_0^{\infty} \mathrm{d} \omega u(\omega, T) \\
                & u(\omega, T)=\frac{\hbar}{\pi^2 c^3} \frac{\omega^3}{e^{\beta \hbar \omega}-1}
                \end{aligned}
            \end{equation}
            此即黑体辐射的普朗克辐射定律。
          \end{deduce}
          \begin{deduce}
            通过经典力学,同样可以得到\(PV=U/3\)(一个方向的辐射能是三分之一总能量)
            
              $$
              \left(\frac{\partial U}{\partial V}\right)_T=3 P=U / V \equiv u(T)
              $$
              $$
              \begin{aligned}
              \left(\frac{\partial U}{\partial V}\right)_T & =T\left(\frac{\partial P}{\partial T}\right)_V-P \\
              u(T) & =\frac{T}{3} \frac{\partial u}{\partial T}-u / 3 \\
              u & =C T^4
              \end{aligned}
              $$
          \end{deduce}
          \section{固体中的声子}
          \begin{claim}
            粒子能量
            \begin{equation}
              \epsilon=\frac{1}{2 m} p^2+\frac{1}{2} m \omega^2 q^2
            \end{equation}
            量子力学当中,一个谐振子的能量为
            \begin{equation}
              \epsilon_n=(n+1/2)\hbar\omega
            \end{equation}
          \end{claim}
          \begin{deduce}
            假设粒子之间相互独立
            \begin{equation}
              Z=\sum e^{-\beta\epsilon_n}=\frac{1}{1-e^{-\beta\hbar\omega}}
            \end{equation}
            \begin{equation}
              \jkuo{\epsilon}=-\pian{\ln Z}{\beta}=\frac{\hbar\omega}{e^{\beta\hbar\omega}-1}
            \end{equation}
            由于有\(3N\)个谐振子,\(U=3N\jkuo{\epsilon}\)
            \begin{equation}
              C_V=\pian{U}{T}=3Nk(\beta\hbar\omega)^2\frac{e^{\beta\hbar\omega}}{\xkuo{e^{\beta\hbar\omega}-1}^2}
            \end{equation}
            与实验不符,因此还需假设声子存在
          \end{deduce}
          \begin{claim}
            声波方程
            \begin{equation}
              \vec \epsilon e^{i(\vec k\cdot\vec r-\omega t)},|\vec k|=\frac\omega v
            \end{equation}
            \(\vec\epsilon\)有三个方向。声子是玻色子。声子能量与\(\dkuo{\omega_i}\)对应
          \end{claim}
          \begin{theorem}
            \hl{德拜理论}
            与之前相同,得到
            \begin{equation}
              f(\omega)=\frac{3V\omega^2}{2\pi^2c^3}
            \end{equation}
            得到最大角频率:
            \begin{equation}
              \begin{aligned}
                & \int_0^{\omega_m} f(\omega) \mathrm{d} \omega=3 N \\
                & \therefore \quad \omega_m=c\left(\frac{6 \pi^2 N}{V}\right)^{1 / 3} \\
                &
                \end{aligned}
            \end{equation}
            与之前相同,得到 
            \begin{equation}
              \ln Z=-\sum_{i=1 }^{3N}\ln(1-e^{-\beta\hbar\omega_i})
            \end{equation}
          \end{theorem}

          


          
          
        \end{document}