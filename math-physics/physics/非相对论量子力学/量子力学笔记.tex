\documentclass[12pt, a4paper, oneside]{ctexbook}
        \usepackage{amsmath, amsthm, amssymb, amsfonts, bm, graphicx, hyperref, mathrsfs}
        \usepackage{tcolorbox}
        \usepackage{tikz, xcolor, environ, xparse, zhnumber}
        
        %设置页眉
        \usepackage{fancyhdr}
        \pagestyle{fancy}
        
        
        
        
        
        
        \usetikzlibrary{shapes, decorations}
        
        %定义颜色
        \definecolor{dedcol}{RGB}{150,150,30}%推论环境的主色
        \definecolor{theocol}{RGB}{40,150,30}%定力环境的主色
        \definecolor{commutecol}{RGB}{30,120,130}
        \definecolor{thrmcol}{RGB}{18,29,80}%默认定理等环境的背景色
        \definecolor{methodcol}{RGB}{49,50,44}
        \definecolor{lawcol}{RGB}{150,30,150}
        \definecolor{thrmedge}{RGB}{12,133,211}%默认定理等环境的边界颜色
        \definecolor{hyperlinkcol}{RGB}{32,112,102}%链接颜色
        \definecolor{hyperfilecol}{RGB}{135,206,235}%文件颜色
        \definecolor{hyperurlcol}{RGB}{3,168,158}%网址颜色
        \definecolor{hypercitecol}{RGB}{150,140,130}%引用颜色
        \definecolor{hlback}{RGB}{207,255,207}%高亮颜色
        \definecolor{opcol}{RGB}{235,125,75}%op颜色
        \definecolor{facecol}{RGB}{122,180,245}%封面颜色
        \definecolor{pscol}{RGB}{44,80,99}%图案颜色
        
        %定义hyper的颜色
        \hypersetup{
          colorlinks=true,
          linkcolor=hyperlinkcol,
          filecolor=hyperfilecol,
          urlcolor=hyperurlcol,
          citecolor=hypercitecol,
        }
        
        %定义字体
        \setCJKfamilyfont{hwxk}{华文行楷}
        \newcommand{\huawenxingkai}{\CJKfamily{hwxk}}
        \setCJKfamilyfont{hwkt}{华文楷体}
        \newcommand{\huawenkaiti}{\CJKfamily{hwkt}}
        \setCJKfamilyfont{hwhp}{华文琥珀}
        \newcommand{\huawenhupo}{\CJKfamily{hwhp}}
        \setCJKfamilyfont{hwls}{华文隶书}
        \newcommand{\huawenlishu}{\CJKfamily{hwls}}
        \setmainfont{华文楷体}
        
        %定义高亮
        \newtcbox{\hlbox}[1][red]{on line, arc = 2pt, outer arc = 0pt,
          colback = hlback, colframe = #1!50!black,
          boxsep = 0pt, left = 1pt, right = 1pt, top = 2pt, bottom = 2pt,
          boxrule = 0pt, bottomrule = 1pt, toprule = 1pt}
        \newcommand{\hl}[1]{\hlbox{#1}}
        \newcommand{\optxt}[1]{\textcolor{opcol}{#1}}
        
        
        
        
        
        %定义公式环境
        \newcommand{\newfancytheoremstyle}[5]{%
          \tikzset{#1/.style={draw=#3, fill=#2,very thick,rectangle,
              rounded corners, inner sep=10pt, inner ysep=20pt}}
          \tikzset{#1title/.style={fill=#3, text=#2}}
          \expandafter\def\csname #1headstyle\endcsname{#4}
          \expandafter\def\csname #1bodystyle\endcsname{#5}
        }
        
        \newfancytheoremstyle{fancythrm}{thrmcol!5}{thrmedge}{\huawenhupo}{\huawenxingkai}
        
        \makeatletter
        \DeclareDocumentCommand{\newfancytheorem}{ O{\@empty} m m m O{fancythrm} }{
          %% 定义计数器
          \ifx#1\@empty
            \newcounter{#2}
          \else
            \newcounter{#2}[#1]
            \numberwithin{#2}{#1}
          \fi
          %% 定义 "newthem" 环境
          \NewEnviron{#2}[1][{}]{%
            \noindent\centering
            \begin{tikzpicture}
              \node[#5] (box){
                \begin{minipage}{0.93\columnwidth}
                  \csname #5bodystyle\endcsname \BODY~##1
                \end{minipage}};
              \node[#5title, right=10pt] at (box.north west){
                {\csname #5headstyle\endcsname #3 \stepcounter{#2}\csname the#2\endcsname\; ##1}};
              \node[#5title, rounded corners] at (box.east) {#4};
            \end{tikzpicture}
          }[\par\vspace{.5\baselineskip}]
        }
        
        
        \makeatother
        
         % 定义各个环境的的样式
         % \newfancytheoremstyle{<name>}{inner color}{outer color}{head style}{body style}
        \newfancytheoremstyle{fancytheo}{theocol!5}{theocol}{\huawenhupo}{\huawenxingkai}
        \newfancytheoremstyle{fancyded}{dedcol!5}{dedcol}{\huawenhupo}{\huawenxingkai}
        \newfancytheoremstyle{fancycom}{commutecol!5}{commutecol}{\huawenhupo}{\huawenxingkai}
        \newfancytheoremstyle{fancymeth}{methodcol!5}{methodcol}{\huawenhupo}{\huawenxingkai}
        \newfancytheoremstyle{fancylaw}{lawcol!5}{lawcol}{\huawenhupo}{\huawenxingkai}
        
         % 定义各个新环境
         % \newfancytheorem[<number within>]{<name>}{<head>}{<symbol>}[<style>]
        \newfancytheorem[chapter]{define}{定义}{$\clubsuit$}
        \newfancytheorem[section]{deduce}{推论}{$\heartsuit$}[fancyded]
        \newfancytheorem[section]{theorem}{定理}{$\spadesuit$}[fancytheo]
        \newfancytheorem[section]{attr}{性质}{}[fancyded]
        \newfancytheorem[chapter]{commute}{对易关系}{}[fancycom]
        \newfancytheorem[section]{intro}{引入}{}[fancyded]
        \newfancytheorem[section]{assmp}{假设}{}[fancylaw]
        \newfancytheorem[chapter]{law}{原理}{}[fancylaw]
        \newfancytheorem{method}{方法}{}[fancymeth]
        
        \title{{\Huge{量子力学笔记}}}
        \author{wave}
        \date{\today}
        \linespread{1.5}
        
        %设置章节标题样式\usepackage[english]{babel}
        \usepackage{blindtext}
        
        \usepackage[sc,compact,explicit]{titlesec} % Titlesec for configuring the header
        
        
        \usepackage{auto-pst-pdf} % Vectorian 装饰图案的 XeTeX 辅助 (见: https://tex.stackexchange.com/questions/253477/how-to-use-psvectorian-with-pdflatex)
        \usepackage{psvectorian} % Vectorian 中的装饰图案
        
        \let\clipbox\relax % PSTricks 已经定义了 \clipbox, 所以要去掉
        \usepackage{adjustbox} % 调整图案大小的
        
        \newcommand{\otherfancydraw}{% 定义图案
        \begin{adjustbox}{max height=0.5\baselineskip}% 根据行距设定高度,自己定
          \raisebox{-0.25\baselineskip}{
          \rotatebox[origin=c]{0}{% 旋转,自己定
            \psvectorian{84}% 图案,编号见 (http://melusine.eu.org/syracuse/pstricks/vectorian/psvectorian.pdf)
          }}%
        \end{adjustbox}%
        }
        
        % 画一条中间为图案的线 (见: https://tex.stackexchange.com/questions/15119/draw-horizontal-line-left-and-right-of-some-text-a-single-line/15122#15122)
        \newcommand*\ruleline[1]{\par\noindent\raisebox{.8ex}{\makebox[\linewidth]{\hrulefill\hspace{1ex}\raisebox{-.8ex}{#1}\hspace{1ex}\hrulefill}}}
        
        \titleformat% Formatting the header
          {\chapter} % command
          [block] % shape - Only managed to get it working with block
          {\normalfont\huawenlishu\huge} % format - Change here as needed
          {\centering 第\zhnum{chapter}章\\ \vspace{-0.6em}} % The Chapter N° label
          {0pt} % sep
          {\centering \ruleline{\otherfancydraw}\\ \vspace{-0.6em} % The horizontal rule
          \centering #1} % And the actual title
        
          \titleformat{\section}[block]{\huawenlishu\Large}{\thesection}{0pt}{\centering #1}

        %更改autoref的形式
        \def\equationautorefname{式}
        \def\footnoteautorefname{脚注}
        \def\itemautorefname{项}
        \def\figureautorefname{图}
        \def\tableautorefname{表}
        \def\appendixautorefname{附录}
        \def\chapterautorefname{章}
        \def\sectionautorefname{小节}
        \def\theoremautorefname{定理}
        



        \newcommand{\com}[2]{\left[#1,#2\right]}
        \newcommand{\xkuo}[1]{\left(#1\right)}
        \newcommand{\dkuo}[1]{\left\lbrace#1\right\rbrace}
        \newcommand{\akuo}[1]{\left[#1\right]}
        \newcommand{\jkuo}[1]{\left\langle#1\right\rangle}
        \newcommand{\wen}[1]{\mbox{#1}}
        \newcommand{\you}{\mbox{又}}
        \newcommand{\dang}{\mbox{当}}
        \newcommand{\yyou}{\mbox{有}}
        \newcommand{\qie}{\mbox{且}}
        \newcommand{\pian}[2]{\frac{\partial #1}{\partial #2}}
        \newcommand{\ppian}[2]{\frac{\partial^2 #1}{\partial #2^2}}
        \newcommand{\dao}[2]{\frac{d#1}{d#2}}
        \newcommand{\ddao}[2]{\frac{d^2#1}{d#2^2}}
        \newcommand{\cen}{^\circ C}
        \newcommand{\fah}{^\circ F}
        \newcommand{\ji}[2]{\int_{#1}^{#2}}
        \newcommand{\qh}[1]{\sum\limits_{#1}}
        \newcommand{\jji}[1]{\iint\limits_{#1}}
        \newcommand{\ppi}{\frac\pi2}
        \newcommand{\ege}{\frac{\sqrt2}{2}}
        \newcommand{\e}[1]{\times10^{#1}}
        \newcommand{\ti}[1]{\textbf{#1}}
        \newcommand{\jdz}[1]{\left|#1\right|}



        %数学分析
        \newcommand{\ya}[4]{\frac{\partial(#1,#2)}{\partial(#3,#4)}}
        \newcommand{\zkya}[4]{\pd #1#3#4\pd #2#4#3-\pd #1#4#3\pd #2#3#4}

        %复变函数
        \newcommand{\wqji}{\int_{-\infty}^{\infty}}

        %热学
        \newcommand{\pd}[3]{\xkuo{\frac{\partial#1}{\partial#2}}_#3}

        %原子物理
        \newcommand{\bra}[1]{\left\langle #1 \right|}
        \newcommand{\ket}[1]{\left| #1 \right\rangle}
        \newcommand{\p}[1]{\partial_{#1}}%对下标的偏导
        \newcommand{\ep}[1]{\epsilon_{#1}}%全反对称张量
        \newcommand{\dt}[1]{\delta_{#1}}%delta张量

        %理论力学
        \newcommand{\keq}[2]{\pian{\mathscr{#1}}{#2}}
        \newcommand{\zkps}[3]{\pian{#1}{q_#3}\pian{#2}{p_#3}-\pian{#1}{p_#3}\pian{#2}{q_#3}}

        %正文
        \newcommand{\sub}[1]{\(_{#1}\)}
        \newcommand{\sps}[1]{\(^{#1}\)}



        \begin{document}
          \renewcommand*{\psvectorianDefaultColor}{pscol}%设定图案颜色
        
          %
            \maketitle
        
            \pagenumbering{roman}
            \setcounter{page}{1}
        
            \begin{center}
                \Huge\huawenlishu{前言}
            \end{center}~\
        
            这是笔记的前言部分.
            ~\\
            \begin{flushright}
                \begin{tabular}{c}
                    何逸阳 \\
                    \today
                \end{tabular}
            \end{flushright}
            \begin{center}
                \Huge\huawenlishu{符号说明}
            \end{center}~\
        
        
            \newpage
            \pagenumbering{alph}
            \setcounter{page}{1}
            \tableofcontents
            \newpage
            \setcounter{page}{1}
            \pagenumbering{arabic}
            \chapter{基本概念}
            \begin{define}
              用\(q\)表示量子系统坐标的集合,用\(dq\)表示其微分的乘积。\(dq\)称为该系统\hl{位形空间}中的一个体积元。
            \end{define}
            \begin{law}
              在给定时刻,一个系统的状态可以用一个确定的\hl{波函数}\(\Psi(q)\)来描述,其模量的平方确定了坐标值的\hl{概率分布}。即对系统进行坐标测量时,测量值处于位行空间的q附近dq体积元当中的概率为\(|\Psi|^2dq\)
            \end{law}
            \begin{law}
              一个系统的\hl{其它测量}的概率由双线性表式
              \begin{equation}
                \iint\Psi(q)\Psi^*(q')\phi(q,q')dqdq'
              \end{equation}
              所确定。

              当\(\phi(q,q')=\delta(q-q_0)\delta(q'-q_0)\)时,就是系统位于坐标\(q_0\)的概率。
            \end{law}
            \begin{law}
              
            \end{law}

            \chapter{有心力场}
            \section{库仑力场}
            \begin{attr}
                \hl{库仑简并}:库仑力场的性质:每一个本征值对\(l\)是简并的,第\(n\)个能级的简并度为
                \begin{equation}
                    \sum_{l=0}^{n-1}(2l+1)=n^2
                \end{equation}
            \end{attr}
            \begin{theorem}
                算符\(\hat A=\frac{\vec r} r-\frac12(\vec p\times\hat l-\hat l\times \vec p)\)是守恒量
            \end{theorem}
            \begin{commute}
                \begin{align}
                    \akuo{\hat l_i,\hat A_j}&=i\epsilon_{ijk}\hat A_k\\
                    \com{\hat A_i}{\hat A_j}&=-2i\hat H\ep{ijk}\hat l_k
                \end{align}
            \end{commute} 

            \section{抛物坐标下库仑场中的运动}
            \begin{intro}
                抛物坐标下的拉普拉斯算子:
                \begin{equation}
                    \Delta=\frac4{\xi+\eta}\akuo{\pian{}{\xi}\xkuo{\xi\pian{}{\xi}}+\pian{}{\eta}\xkuo{\eta\pian{}{\eta}}}+\frac{1}{\xi\eta}\ppian{}{\varphi}
                \end{equation}
            \end{intro}
            \begin{deduce}
                库仑力场
                \begin{equation}
                    U=-\frac1r=-\frac{2}{\xi+\eta}
                \end{equation}
                中的单粒子薛定谔方程为:
                \begin{align}
                    0&=\frac12\Delta \psi+\hat H\psi\\
                    \label{eqn:pkl}&=\frac2{\xi+\eta}\akuo{\pian{}{\xi}\xkuo{\xi\pian{\psi}{\xi}}+\pian{}{\eta}\xkuo{\eta\pian{\psi}{\eta}}}+\frac{1}{2\xi\eta}\ppian{\psi}{\varphi}+\xkuo{E+\frac{2}{\xi+\eta}}\psi
                \end{align}
            \end{deduce}
            \begin{deduce}
                
                对\autoref{eqn:pkl}利用分离变量法可得
                \begin{equation}
                    \label{eqn:pklfl}
                    \begin{cases}
                        f_\varphi(\varphi)=e^{im=\varphi}\\
                        \dao{}{\xi}\xkuo{\xi\dao{f_\xi}{\xi}}+\akuo{\frac12E\xi-\frac{m^2}{4\xi}+\beta_\xi}f_\xi=0\\
                        \dao{}{\eta}\xkuo{\eta\dao{f_\eta}{\eta}}+\akuo{\frac12E\eta-\frac{m^2}{4\eta}+\beta_\eta}f_\eta=0
                    \end{cases}
                \end{equation}
                其中\(m\in\mathbb Z,\,\beta_\xi+\beta_\eta=1\)
            \end{deduce}
            \begin{theorem}
                由\autoref{eqn:pklfl}可得离散谱的定态由三个整数确定,分别称为\hl{抛物量子数}\(n_1,n_2\)和\hl{磁量子数}\(m\),其\hl{主量子数的表达式}为
                \begin{equation}
                    n=n_1+n_2+|m|+1
                \end{equation}
            \end{theorem}
            \begin{deduce}
                归一化的波函数为:
                \begin{equation}
                    \psi_{n_1n_2m}=\frac{\sqrt[]{2}}{n^2}f_{n_1m}\xkuo{\frac{\xi}{n}}f_{n_2m}\xkuo{\frac{\eta}{n}}\frac{e^{im\varphi}}{\sqrt[]{2\pi}}
                \end{equation}
                其中
                \begin{equation}
                    f_{pm}(\rho)=\frac{1}{|m|!}\sqrt[]{\frac{(p+|m|)!}{p!}}F(-p,|m|+1,\rho)e^{-\frac12\rho}\rho^{\frac12|m|}
                \end{equation}
            \end{deduce}
            \chapter{微扰论}
            \begin{method}
              由于薛定谔方程很难得到精确解,但忽略一些小量即可得到很好的解,于是可以通过\hl{微扰论}的方法把问题分为两步:
              \begin{enumerate}
                \item 求出忽略小量的简化问题的精确解
                \item 计算忽略小量引起的误差
              \end{enumerate}
            \end{method}
            \section{与时间无关的微扰}
            \begin{assmp}
              设哈密顿量呈以下形式:
              \begin{equation}
                \hat H=\hat H_0+\hat V
              \end{equation}
              其中算符\(\hat V\)代表微扰项,而\(\hat H_0\)代表未受扰项,其解已知:
              \begin{equation}
                \hat H_0\psi^{(0)}=E^{(0)}\psi^{(0)}
              \end{equation}
              假设\(\hat H_0\)无简并。
            \end{assmp}
            \begin{theorem}
              设\(\psi=\sum\limits_mc_m\psi_m^{(0)}\)可得
              \begin{equation}\label{eqn:disturb}
                \xkuo{E-E_k^{(0)}}c_k=\sum_mV_{km}c_m
              \end{equation}
              将第\(n\)个本征态的本征值的一级近似
              \begin{equation}
                E_n=E_n^{(0)}+E_n^{(1)}
              \end{equation}
              代入\autoref{eqn:disturb}可得
              \begin{equation}
                E_n^{(1)}=V_{nn}=\int\psi_n^{(0)*}\hat V\psi_n^{(0)}dq
              \end{equation}
              同时,其本征函数的一级近似
              \begin{equation}
                \psi_n=\psi_n^{(0)}+\psi_n^{(1)}
              \end{equation}
              代入可得
              \begin{equation}
                \psi_n^{(1)}=\sum_{m\neq n}\frac{V_{mn}}{E_n^{(0)}-E_m^{(0)}}\psi_m^{(0)}
              \end{equation}
              因此,要使得微扰项足够小(一阶小量)需要满足的条件为:
              \begin{equation}
                |V_{mn}|<<|E_n^{(0)}-E_m^{(0)}|
              \end{equation}
              进而可以算出物理量\(f\)的微扰:
              \begin{equation}
                f_{nm}=f_{nm}^{(0)}+\sum_{k\neq n}\frac{V_{nk}f_{km}^{(0)}}{E_n^{(0)}-E_k^{(0)}}+\sum_{k\neq m}\frac{V_{km}f_{nk}^{(0)}}{E_m^{(0)}-E_k^{(0)}}
              \end{equation}
              更高阶的近似同理
            \end{theorem}

            \section{久期方程}
            \begin{assmp}
              假设\(\hat H_0\)具有简并,而\(\psi_{n'},n'\in\Lambda \)代表能级\(E_n^{(0)}\)下的一套本征函数。
            \end{assmp}
            \begin{theorem}
              通过同样的方法可以得到
              \begin{equation}
                \sum_{n,n'\in \Lambda }\xkuo{V_{nn'}-E^{(1)}\delta_{nn'}}c_{n'}^{(0)}=0
              \end{equation}
              若要此方程有非零解,则需要
              \begin{equation}\label{eqn:seculareqn}
                |V_{nn'}-E^{(1)}\delta_{nn'}|=0
              \end{equation}
              \autoref{eqn:seculareqn}称为\hl{久期方程}。
            \end{theorem}
            \section{与时间有关的微扰}
            \begin{assmp}
              若微扰项含时间,而未扰系统的定态波函数为\(\Psi_k^{(0)}\),设受扰方程的解为
              \begin{equation}\label{eqn:timedistsol}
                \Psi=\sum_ka_k(t)\Psi_k^{(0)}
              \end{equation}
            \end{assmp}

            \begin{theorem}
              将\autoref{eqn:timedistsol}代入薛定谔方程后,可求得
              \begin{equation}
                i\hslash\dao{a_m}{t}=\sum_kV_{mk}(t)a_k
              \end{equation}
              其中
              \begin{equation}
                V_{mk}(t)=\int\Psi_m^{(0)*}\hat V\Psi_k^{(0)}dq=V_{mk}e^{i\omega_{mk}t}
              \end{equation}
              \begin{equation}
                \omega_{mk}=\frac{E_m^{(0)}-E_k^{(0)}}{\hslash}
              \end{equation}
              当未受扰
            \end{theorem}

            \section{有限时间作用下的跃迁}
            \begin{assmp}
              假定微扰\(V(t)\)作用于有限时间内(或者当\(t\rightarrow\pm \infty\)时,\(V(t)\)足够快地趋于零)。
            \end{assmp}
            \begin{assmp}
              假定微扰作用前,系统处于离散谱的第n个态中,随后由\(\Psi=\sum_ka_{kn}\Psi_k^{(0)}\)确定。
            \end{assmp}
            \section{周期微扰作用下的跃迁}
            朗道P140
            \begin{assmp}
              设在t=0时刻,系统处于离散谱的第i个定态,且周期微扰的
            \end{assmp}
            \section{连续谱中的跃迁}
            \begin{assmp}
              
            \end{assmp}

            \chapter{自旋}
            \begin{define}
              由于粒子内部确定的运动,粒子具有一定的量子化的角动量\(L\),可以有\(2L+1\)种空间取向。因此每个基本粒子被赋予一个与其空间运动无关的\hl{内禀}角动量,称为\hl{自旋}。
            \end{define}

            \begin{deduce}
              一个粒子的\hl{总角动量}等于其轨道角动量\(\boldsymbol{j}=\boldsymbol{l}+\boldsymbol{s}\),其加法与矢量加法模型一致。它满足之前推导得到的所有两个角动量耦合的公式。
            \end{deduce}

            \section{自旋算符}
            之后的推导将不包含任何坐标,只包含自旋变量\(\sigma\)。
            \begin{define}
              作用在\(\sigma\)的函数上的算符可以表成一个\((2s+1)\)行\((2s+1)\)列的矩阵。自旋算符本身的矩阵元素与之前求得的矩阵\(\hat L_x,\hat L_y,\hat L_z\)

              在自旋为\(\frac12\)的特殊情形下,矩阵是\(2\times2\)的:
              \begin{equation}\label{eqn:spinOperator}
                \hat {\boldsymbol{s}}=\frac12\hat{\boldsymbol{\sigma}}
              \end{equation}\label{eqn:pauliMatrix}
              其中\(\hat{\boldsymbol{\sigma}}\)为\hl{泡利矩阵}:
              \begin{equation}
                \hat{\sigma}_x=\begin{pmatrix}
                  0&1\\1&0
                \end{pmatrix},\,\hat{\sigma}_y=\begin{pmatrix}
                  0&-i\\i&0
                \end{pmatrix},\,\hat\sigma_z=\begin{pmatrix}
                  1&0\\0&-1
                \end{pmatrix}
              \end{equation}
            \end{define}
            \begin{attr}
              泡利矩阵相乘:
              \begin{equation}\label{eqn:pauliMatrix-times}
                \hat\sigma_i\hat\sigma_j=\delta_{ij}+i\varepsilon_{ijk}\sigma_k
              \end{equation}
              由此可得\begin{equation}\label{eqn:pauliMatrix-anticommute}
                \hat\sigma_i\hat\sigma_k+\hat\sigma_k\hat\sigma_i=2\delta_{ik}
              \end{equation}
            \end{attr}
            \begin{attr}
              泡利矢量的性质:
              \begin{equation}\label{eqn:pauliVector-times}
                \hat{\boldsymbol{\sigma}}^2=3,\quad(\hat{\boldsymbol{\sigma}}\cdot\hat{\boldsymbol{a}})(\hat{\boldsymbol{\sigma}}\cdot\hat{\boldsymbol{b}})=\boldsymbol{a}\cdot\boldsymbol{b}+i\hat{\boldsymbol{\sigma}}\cdot\boldsymbol{a}\times\boldsymbol{b}
              \end{equation}
            \end{attr}
            \begin{deduce}
              假定绕z轴转动无限小角度\(\delta\phi\),这种转动的算符,可通过角动量算符表成\(1+i\delta\phi\delta\cdot\hat s_z\),于是有\(d\Psi/d\phi=i\sigma\Psi(\sigma)\),所以转过一个有限角\(\phi\)后,有 
              \begin{equation}
                \Psi'(\sigma)=\Psi(\sigma)e^{i\sigma\phi}
              \end{equation}
            \end{deduce}

            \section{旋量}
            \begin{deduce}
              自旋为0时,波函数只有一个分量\(\psi(0)\),即标量。

              自旋为1/2时,波函数有两个分量\(\psi=\begin{pmatrix}
                \psi^1\\\psi^2
              \end{pmatrix}\),称为一个\hl{旋量}
              
            \end{deduce}
            \begin{deduce}
              
              在坐标系的转动中,旋量做线性变换:
              \begin{equation}
                \psi'=\hat U\psi
              \end{equation}
              考虑粒子概率\(\psi^1\phi^{1*}-\psi^2\phi^{2*}\),其应在变换下不变,于是我们有
              \begin{equation}
                \hat U^+=\hat U^{-1}
              \end{equation}
            \end{deduce}

            \chapter{准经典情形}
            \section{准经典下的波函数}
            \begin{deduce}
              设波函数
              \begin{equation}\label{eqn:wavefun-sigma}
                \psi=e^{\xkuo{i/\hslash}\sigma}
              \end{equation}
              代入薛定谔方程可得
              \begin{equation}\label{eqn:sigma}
                \sum_a\frac{1}{2m_a}\xkuo{\nabla_a\sigma}^2-\sum_a\frac{i\hslash}{2m_a}\Delta_a\sigma=E-U
              \end{equation}
              在准经典近似下展开,\(\sigma=\sigma_0+\xkuo{\frac\hslash i}\sigma_1+\xkuo{\frac\hslash i}^2\sigma_2+\cdots\)
              
              一维情况下,\autoref{eqn:sigma}变为
              \begin{equation}
                \frac1{2m}\sigma'^2-\frac{i\hslash}{2m}\sigma''=E-U(x)
              \end{equation}
              零级近似下,\(\sigma=\sigma_0\)代入并忽略\(\hslash\)项得 
              \begin{equation}\label{eqn:sigma0}
                \frac{1}{2m}\sigma'^2_0=E-U(x)
              \end{equation}
              由此可见\(\sigma'_0\)刚好是经典动量。
            \end{deduce}
            \begin{define}
              \autoref{eqn:sigma0}成立的条件是第二项远小于第一项(由于是微分方程,所以这个条件不充分):
              \begin{equation}\label{eqn:quasi-classical-criterion}
                \left|\dao{}{x}\xkuo{\frac{\hslash}{\sigma_0'}}\right|\ll1
              \end{equation}
              由于此近似下\(\sigma'_0=p\),上式可化为
              \begin{equation}
                \left|\dao{}{x}\xkuo{\frac{\lambda}{2\pi}}\right|\ll1
              \end{equation}
              此即\hl{准经典条件}
            \end{define}
            \begin{deduce}
              注意到\(\dao px=\frac{mF}p\), 代入\autoref{eqn:quasi-classical-criterion}可以得到
              \begin{equation}
                \frac{m\hslash|F|}{p^3}\ll1
              \end{equation}
              其中\(F=-dU/dx\)为粒子在外场所受经典力。

              由此可知,当粒子位于动量太小(即速度接近于0,称为\hl{回点\label{concept:backpoint}})的点时,准经典近似就不适用了。
            \end{deduce}
            \begin{deduce}
              将\autoref{eqn:sigma}保留到\(\hslash\)的一阶项并代入零阶的\(\sigma_0\)可以得到
              \begin{equation}
                \sigma_1=-\frac12\ln p
              \end{equation}
            \end{deduce}
            \begin{deduce}
              将\(\sigma_0\)和\(\sigma_1\)代入\autoref{eqn:wavefun-sigma}中得到
              \begin{equation}
                \psi=\frac{C_1}{\sqrt p}\exp\xkuo{\frac i\hslash\int pdx}+\frac{C_2}{\sqrt p}\exp\xkuo{-\frac i\hslash\int pdx}
              \end{equation}
            \end{deduce}
            \section{准经典下的边界条件}
            \begin{deduce}
              设\(x=a\)为一个回点,满足\(U(a)=E\),其右侧均有\(U(x)>E\),为经典禁区,则离回点足够远处的两边有
              \begin{equation}
                \begin{cases}
                  \psi=\frac{C}{w\sqrt{|p|}}\exp\xkuo{-\frac1\hslash\jdz{\ji{a}{x}pdx}}, &x>a\\
                  \psi=\frac{C_1}{\sqrt p}\exp\xkuo{\frac i\hslash\ji ax pdx}+\frac{C_2}{\sqrt p}\exp\xkuo{-\frac i\hslash\ji ax pdx},&x<a
                \end{cases}
              \end{equation}
              x在a附近的行为就无法用准经典近似,因此需要考虑薛定谔方程的精确解。对于小量\(|x-a|\),一阶展开有:
              \begin{equation}
                E-U(x)\approx F_0(x-a),\, F_0=-\left.\dao Ux\right|_{x=a}
              \end{equation}
              即线性势的精确解,通过渐进式算得常数之间的关系
              
            \end{deduce}

            \begin{deduce}
              对应规则可以写为
              \begin{equation}\label{eqn:edge}
                \begin{cases}
                  U(x)>E:\quad \frac{C}{2\sqrt{|p|}}\exp\dkuo{-\frac1\hslash\jdz{\ji ax pdx}}\\
                  U(x)<E:\quad \frac{C}{\sqrt{p}}\cos\dkuo{\frac1\hslash\jdz{\ji ax pdx}-\frac\pi4}
                \end{cases}
              \end{equation}
            \end{deduce}

            \section{波尔-索末菲量子化规则}
            \begin{deduce}
              考虑一个一维势阱,其经典通区为\(b\leqslant x\leqslant a\)介于两个回点之间。通过\autoref{eqn:edge}可以得到两个式子。为了使这两个式子兼容,它们的相位差别必须等于一个整数,于是我们得到
              \begin{equation}\label{eqn:bore-somerfeld}
                \frac{1}{2\pi\hslash}\oint pdx=n+\frac12
              \end{equation}
              将\hl{浸渐不变量}\(I=\frac1{2\pi}\oint p dx\)代入,上式可写成
              \begin{equation}
                I(E)=\hslash\xkuo{n+\frac12}
              \end{equation}
              又因为积分\(\oint pdx\)等于经典粒子相空间中闭合相迹所包围的面积,由上式它等于\(2\pi\hslash\xkuo{n+\frac12}\)。公式中可以看出能量不高于此相迹的量子态数n就等于面积除以\(2\pi\hslash\),于是相空间中单位面积所包含的不高于相迹能量的量子数为:
              \begin{equation}
                n(E)=\frac{\Delta p\Delta x}{2\pi\hslash}
              \end{equation}
            \end{deduce}
            \begin{deduce}
              在准经典中,n很大,\autoref{eqn:edge}中的余弦函数是剧变的,所以可以把对余弦函数的平方的积分改成它的平均值的平方,同时可以把积分限制在\([a,b]\),因此归一化得到
              \begin{equation*}
                \int|\psi|^2dx\approx\frac12C^2\ji ba\frac{dx}{p(x)}=\frac{\pi C^2}{2m\omega}=1
              \end{equation*}
              其中\(\omega=2\pi/T\),为经典粒子周期(注意\(p(x)=mdx/dt\)),于是归一化的波函数为
              \begin{equation}
                \psi=\sqrt{\frac{2\omega}{\pi v}}\cos\xkuo{\frac1\hslash\ji bx pdx-\frac14\pi}
              \end{equation}
            \end{deduce}
            \begin{deduce}
              对于相邻两个能级之间的间距,可以从\autoref{eqn:bore-somerfeld}出发。准经典情况下,由于能量差\(\Delta E\)比能量本身小得多,因此可以得到
              \begin{equation*}
                \oint \pian pE\Delta Edx=2\pi\hslash
              \end{equation*}
              同时由于\(\pian Ep=v\),因此有
              \begin{equation*}
                \oint \pian pEdx=\oint \frac{dx}{v}=T
              \end{equation*}
              于是我们得到
              \begin{equation}\label{eqn:sc-DeltaE}
                \Delta E=\hslash\omega
              \end{equation}
            \end{deduce}
            \begin{deduce}
              对于任意物理量在波包的一组基矢\(\Psi=\sum\limits_{n}a_n\Psi_n\)中的矩阵元,其经典量应当对应于其平均值: 
              \begin{equation*}
                \bar f=\int\Psi^*\hat f\Psi dx=\sum_n\sum_ma^*_ma_nf_{mn}e^{i\omega_{mn}t}
              \end{equation*}
              将m,n的求和用n和\(s=m-n\)来替代,我们得到
              \begin{equation*}
                \bar f=\sum_n\sum_s a^*_{n+s}a_nf_{n+s,n}e^{is\omega t}
              \end{equation*}
            \end{deduce}

            \section{有心力场中的准经典运动}
            \begin{deduce}
              对于球系的微分方程,考虑角部。当m=0时,其正比于勒让德多项式\(P_l(\cos \theta)\),满足勒让德方程。做替换\(P_l(\cos\theta)=\frac{\chi(\theta)}{\sqrt{\sin\theta}}\),方程化为
              \begin{equation}
                \chi''+\akuo{\xkuo{l+\frac12}^2+\frac{1}{4\sin^2\theta}}\chi=0
              \end{equation}
              这与一维薛定谔方程具有相同形式,其德布罗意波长为
              \begin{equation*}
                \lambda = 2\pi\akuo{\xkuo{l+\frac12}^2+\frac1{4\sin^2\theta}}^{-\frac12}
              \end{equation*}
              根据准经典条件\autoref{eqn:quasi-classical-criterion},得到下列不等式:
              \begin{equation}
                \theta l>>1,\,(\pi-\theta)l>>1
              \end{equation}
              此时方括号中的\(\frac1{4\sin^2\theta}\)项可以略去,我们得到二阶常微分方程。
            \end{deduce}

            


            



        
        \end{document}