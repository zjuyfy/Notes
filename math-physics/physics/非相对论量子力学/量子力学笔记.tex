\documentclass[12pt, a4paper, oneside]{ctexbook}
        \usepackage{amsmath, amsthm, amssymb, amsfonts, bm, graphicx, hyperref, mathrsfs}
        \usepackage{tcolorbox}
        \usepackage{tikz, xcolor, environ, xparse, zhnumber}
        
        %设置页眉
        \usepackage{fancyhdr}
        \pagestyle{fancy}
        
        
        
        
        
        
        \usetikzlibrary{shapes, decorations}
        
        %定义颜色
        \definecolor{dedcol}{RGB}{150,150,30}%推论环境的主色
        \definecolor{theocol}{RGB}{40,150,30}%定力环境的主色
        \definecolor{commutecol}{RGB}{30,120,130}
        \definecolor{thrmcol}{RGB}{18,29,80}%默认定理等环境的背景色
        \definecolor{methodcol}{RGB}{49,50,44}
        \definecolor{lawcol}{RGB}{150,30,150}
        \definecolor{thrmedge}{RGB}{12,133,211}%默认定理等环境的边界颜色
        \definecolor{hyperlinkcol}{RGB}{32,112,102}%链接颜色
        \definecolor{hyperfilecol}{RGB}{135,206,235}%文件颜色
        \definecolor{hyperurlcol}{RGB}{3,168,158}%网址颜色
        \definecolor{hypercitecol}{RGB}{150,140,130}%引用颜色
        \definecolor{hlback}{RGB}{207,255,207}%高亮颜色
        \definecolor{opcol}{RGB}{235,125,75}%op颜色
        \definecolor{facecol}{RGB}{122,180,245}%封面颜色
        \definecolor{pscol}{RGB}{44,80,99}%图案颜色
        
        %定义hyper的颜色
        \hypersetup{
          colorlinks=true,
          linkcolor=hyperlinkcol,
          filecolor=hyperfilecol,
          urlcolor=hyperurlcol,
          citecolor=hypercitecol,
        }
        
        %定义字体
        \setCJKfamilyfont{hwxk}{华文行楷}
        \newcommand{\huawenxingkai}{\CJKfamily{hwxk}}
        \setCJKfamilyfont{hwkt}{华文楷体}
        \newcommand{\huawenkaiti}{\CJKfamily{hwkt}}
        \setCJKfamilyfont{hwhp}{华文琥珀}
        \newcommand{\huawenhupo}{\CJKfamily{hwhp}}
        \setCJKfamilyfont{hwls}{华文隶书}
        \newcommand{\huawenlishu}{\CJKfamily{hwls}}
        \setmainfont{华文楷体}
        
        %定义高亮
        \newtcbox{\hlbox}[1][red]{on line, arc = 2pt, outer arc = 0pt,
          colback = hlback, colframe = #1!50!black,
          boxsep = 0pt, left = 1pt, right = 1pt, top = 2pt, bottom = 2pt,
          boxrule = 0pt, bottomrule = 1pt, toprule = 1pt}
        \newcommand{\hl}[1]{\hlbox{#1}}
        \newcommand{\optxt}[1]{\textcolor{opcol}{#1}}
        
        
        
        
        
        %定义公式环境
        \newcommand{\newfancytheoremstyle}[5]{%
          \tikzset{#1/.style={draw=#3, fill=#2,very thick,rectangle,
              rounded corners, inner sep=10pt, inner ysep=20pt}}
          \tikzset{#1title/.style={fill=#3, text=#2}}
          \expandafter\def\csname #1headstyle\endcsname{#4}
          \expandafter\def\csname #1bodystyle\endcsname{#5}
        }
        
        \newfancytheoremstyle{fancythrm}{thrmcol!5}{thrmedge}{\huawenhupo}{\huawenxingkai}
        
        \makeatletter
        \DeclareDocumentCommand{\newfancytheorem}{ O{\@empty} m m m O{fancythrm} }{
          %% 定义计数器
          \ifx#1\@empty
            \newcounter{#2}
          \else
            \newcounter{#2}[#1]
            \numberwithin{#2}{#1}
          \fi
          %% 定义 "newthem" 环境
          \NewEnviron{#2}[1][{}]{%
            \noindent\centering
            \begin{tikzpicture}
              \node[#5] (box){
                \begin{minipage}{0.93\columnwidth}
                  \csname #5bodystyle\endcsname \BODY~##1
                \end{minipage}};
              \node[#5title, right=10pt] at (box.north west){
                {\csname #5headstyle\endcsname #3 \stepcounter{#2}\csname the#2\endcsname\; ##1}};
              \node[#5title, rounded corners] at (box.east) {#4};
            \end{tikzpicture}
          }[\par\vspace{.5\baselineskip}]
        }
        
        
        \makeatother
        
         % 定义各个环境的的样式
         % \newfancytheoremstyle{<name>}{inner color}{outer color}{head style}{body style}
        \newfancytheoremstyle{fancytheo}{theocol!5}{theocol}{\huawenhupo}{\huawenxingkai}
        \newfancytheoremstyle{fancyded}{dedcol!5}{dedcol}{\huawenhupo}{\huawenxingkai}
        \newfancytheoremstyle{fancycom}{commutecol!5}{commutecol}{\huawenhupo}{\huawenxingkai}
        \newfancytheoremstyle{fancymeth}{methodcol!5}{methodcol}{\huawenhupo}{\huawenxingkai}
        \newfancytheoremstyle{fancylaw}{lawcol!5}{lawcol}{\huawenhupo}{\huawenxingkai}
        
         % 定义各个新环境
         % \newfancytheorem[<number within>]{<name>}{<head>}{<symbol>}[<style>]
        \newfancytheorem[chapter]{define}{定义}{$\clubsuit$}
        \newfancytheorem[section]{deduce}{推论}{$\heartsuit$}[fancyded]
        \newfancytheorem[section]{theorem}{定理}{$\spadesuit$}[fancytheo]
        \newfancytheorem[section]{attr}{性质}{}[fancyded]
        \newfancytheorem[chapter]{commute}{对易关系}{}[fancycom]
        \newfancytheorem[section]{intro}{引入}{}[fancyded]
        \newfancytheorem[section]{assmp}{假设}{}[fancylaw]
        \newfancytheorem[chapter]{law}{原理}{}[fancylaw]
        \newfancytheorem{method}{方法}{}[fancymeth]
        
        \title{{\Huge{量子力学笔记}}}
        \author{wave}
        \date{\today}
        \linespread{1.5}
        
        %设置章节标题样式\usepackage[english]{babel}
        \usepackage{blindtext}
        
        \usepackage[sc,compact,explicit]{titlesec} % Titlesec for configuring the header
        
        
        \usepackage{auto-pst-pdf} % Vectorian 装饰图案的 XeTeX 辅助 (见: https://tex.stackexchange.com/questions/253477/how-to-use-psvectorian-with-pdflatex)
        \usepackage{psvectorian} % Vectorian 中的装饰图案
        
        \let\clipbox\relax % PSTricks 已经定义了 \clipbox, 所以要去掉
        \usepackage{adjustbox} % 调整图案大小的
        
        \newcommand{\otherfancydraw}{% 定义图案
        \begin{adjustbox}{max height=0.5\baselineskip}% 根据行距设定高度,自己定
          \raisebox{-0.25\baselineskip}{
          \rotatebox[origin=c]{0}{% 旋转,自己定
            \psvectorian{84}% 图案,编号见 (http://melusine.eu.org/syracuse/pstricks/vectorian/psvectorian.pdf)
          }}%
        \end{adjustbox}%
        }
        
        % 画一条中间为图案的线 (见: https://tex.stackexchange.com/questions/15119/draw-horizontal-line-left-and-right-of-some-text-a-single-line/15122#15122)
        \newcommand*\ruleline[1]{\par\noindent\raisebox{.8ex}{\makebox[\linewidth]{\hrulefill\hspace{1ex}\raisebox{-.8ex}{#1}\hspace{1ex}\hrulefill}}}
        
        \titleformat% Formatting the header
          {\chapter} % command
          [block] % shape - Only managed to get it working with block
          {\normalfont\huawenlishu\huge} % format - Change here as needed
          {\centering 第\zhnum{chapter}章\\ \vspace{-0.6em}} % The Chapter N° label
          {0pt} % sep
          {\centering \ruleline{\otherfancydraw}\\ \vspace{-0.6em} % The horizontal rule
          \centering #1} % And the actual title
        
          \titleformat{\section}[block]{\huawenlishu\Large}{\thesection}{0pt}{\centering #1}

        %更改autoref的形式
        \def\equationautorefname{式}
        \def\footnoteautorefname{脚注}
        \def\itemautorefname{项}
        \def\figureautorefname{图}
        \def\tableautorefname{表}
        \def\appendixautorefname{附录}
        \def\chapterautorefname{章}
        \def\sectionautorefname{小节}
        \def\theoremautorefname{定理}
        



        \newcommand{\com}[2]{\left[#1,#2\right]}
        \newcommand{\xkuo}[1]{\left(#1\right)}
        \newcommand{\dkuo}[1]{\left\lbrace#1\right\rbrace}
        \newcommand{\akuo}[1]{\left[#1\right]}
        \newcommand{\jkuo}[1]{\left\langle#1\right\rangle}
        \newcommand{\wen}[1]{\mbox{#1}}
        \newcommand{\you}{\mbox{又}}
        \newcommand{\dang}{\mbox{当}}
        \newcommand{\yyou}{\mbox{有}}
        \newcommand{\qie}{\mbox{且}}
        \newcommand{\pian}[2]{\frac{\partial #1}{\partial #2}}
        \newcommand{\ppian}[2]{\frac{\partial^2 #1}{\partial #2^2}}
        \newcommand{\dao}[2]{\frac{d#1}{d#2}}
        \newcommand{\ddao}[2]{\frac{d^2#1}{d#2^2}}
        \newcommand{\cen}{^\circ C}
        \newcommand{\fah}{^\circ F}
        \newcommand{\ji}[2]{\int_{#1}^{#2}}
        \newcommand{\qh}[1]{\sum\limits_{#1}}
        \newcommand{\jji}[1]{\iint\limits_{#1}}
        \newcommand{\ppi}{\frac\pi2}
        \newcommand{\ege}{\frac{\sqrt2}{2}}
        \newcommand{\e}[1]{\times10^{#1}}
        \newcommand{\ti}[1]{\textbf{#1}}



        %数学分析
        \newcommand{\ya}[4]{\frac{\partial(#1,#2)}{\partial(#3,#4)}}
        \newcommand{\zkya}[4]{\pd #1#3#4\pd #2#4#3-\pd #1#4#3\pd #2#3#4}

        %复变函数
        \newcommand{\wqji}{\int_{-\infty}^{\infty}}

        %热学
        \newcommand{\pd}[3]{\xkuo{\frac{\partial#1}{\partial#2}}_#3}

        %原子物理
        \newcommand{\bra}[1]{\left\langle #1 \right|}
        \newcommand{\ket}[1]{\left| #1 \right\rangle}
        \newcommand{\p}[1]{\partial_{#1}}%对下标的偏导
        \newcommand{\ep}[1]{\epsilon_{#1}}%全反对称张量
        \newcommand{\dt}[1]{\delta_{#1}}%delta张量

        %理论力学
        \newcommand{\keq}[2]{\pian{\mathscr{#1}}{#2}}
        \newcommand{\zkps}[3]{\pian{#1}{q_#3}\pian{#2}{p_#3}-\pian{#1}{p_#3}\pian{#2}{q_#3}}

        %正文
        \newcommand{\sub}[1]{\(_{#1}\)}
        \newcommand{\sps}[1]{\(^{#1}\)}



        \begin{document}
          \renewcommand*{\psvectorianDefaultColor}{pscol}%设定图案颜色
        
          %
            \maketitle
        
            \pagenumbering{roman}
            \setcounter{page}{1}
        
            \begin{center}
                \Huge\huawenlishu{前言}
            \end{center}~\
        
            这是笔记的前言部分.
            ~\\
            \begin{flushright}
                \begin{tabular}{c}
                    何逸阳 \\
                    \today
                \end{tabular}
            \end{flushright}
            \begin{center}
                \Huge\huawenlishu{符号说明}
            \end{center}~\
        
        
            \newpage
            \pagenumbering{alph}
            \setcounter{page}{1}
            \tableofcontents
            \newpage
            \setcounter{page}{1}
            \pagenumbering{arabic}
            \chapter{基本概念}
            \begin{define}
              用\(q\)表示量子系统坐标的集合,用\(dq\)表示其微分的乘积。\(dq\)称为该系统\hl{位形空间}中的一个体积元。
            \end{define}
            \begin{law}
              在给定时刻,一个系统的状态可以用一个确定的\hl{波函数}\(\Psi(q)\)来描述,其模量的平方确定了坐标值的\hl{概率分布}。即对系统进行坐标测量时,测量值处于位行空间的q附近dq体积元当中的概率为\(|\Psi|^2dq\)
            \end{law}
            \begin{law}
              一个系统的\hl{其它测量}的概率由双线性表式
              \begin{equation}
                \iint\Psi(q)\Psi^*(q')\phi(q,q')dqdq'
              \end{equation}
              所确定。

              当\(\phi(q,q')=\delta(q-q_0)\delta(q'-q_0)\)时,就是系统位于坐标\(q_0\)的概率。
            \end{law}
            \begin{law}
              \hl{状态叠加原理}
              设一种测量在给定两态\(\Psi_1,\Psi_2\)中测量分别得到肯定的结果1和2,那么
              \begin{enumerate}
                \item 在任意具有\(c_1\Psi_1+c_2\Psi_2\)形式的态(称为\hl{线性叠加})中,测量结果要么是1要么是2.
                \item 若以上两个态对时间的依赖关系已知,则其任意线性叠加给出了叠加态的可能的时间依赖关系。
              \end{enumerate}
              因此,波函数满足的一切方程必须对\(\Psi\)保持线性。
            \end{law}
            \section{算符}
            \begin{define}
              一个给定物理量所能取的数值称为\hl{本征值},这些数值的集合称为该量的\hl{值谱}。连续的值谱称为\hl{连续谱},离散的称为\hl{离散谱}
            \end{define}
            \begin{assmp}
              本节接下来所有推论针对离散谱。
            \end{assmp}
            \begin{deduce}
              任何波函数可以用任意物理量的一套本征函数来展开\(\Psi=\sum_na_n\Psi_n\),这组函数称为一个\hl{完备组(封闭组)}。这些函数相互\hl{正交}。并且
              \begin{equation}\label{eqn:an}
                a_n=\int\Psi\Psi_n^*dq
              \end{equation}
            \end{deduce}
            \begin{define}
              定义物理量的\hl{平均值}
              \begin{equation}
                \bar f=\sum_nf_n|a_n|^2
              \end{equation}
              其中\(f_n\)是本征值。
            \end{define}
            \begin{define}
              引入\hl{算符}\(\hat f\)为一种作用于波函数得到波函数的映射,定义为使得
              \begin{equation}
                \bar f=\int \Psi^*(\hat f\Psi)dq
              \end{equation}
            \end{define}
            \begin{deduce}
              对比可得
              \begin{equation}\label{eqn:fPsi}
              (\hat f\Psi)=\sum_na_nf_n\Psi_n
              \end{equation}
              物理量\(f\)的本征函数为以下方程的解:
              \begin{equation}
                \hat f\Psi=f\Psi
              \end{equation}
            \end{deduce}
            \begin{define}
              将\autoref{eqn:an}代入\autoref{eqn:fPsi}可得
              \begin{equation}
                (\hat f\Psi)=\int K(q,q')\Psi(q')dq'
              \end{equation}
              其中\(K(q,q')\)称为该算符的\hl{核},定义为 
              \begin{equation}
                K(q,q')=\sum_nf_n\Psi_n^*(q')\Psi_n(q)
              \end{equation}
            \end{define}
            \begin{deduce}
              由于物理量的本征值及平均值应为实数,这就限制了算符\(\hat f\)为厄米算符:
              \begin{equation}
                \hat f=\hat f^\dagger
              \end{equation}
              物理量的不同本征值的本征函数互相正交。
            \end{deduce}
            \section{算符的加法和乘法}
            \begin{law}
              设\(\hat f\)和\(\hat g\)是两个物理量。若两个物理量可以同时测量,两个算符就有共同本征态,算符\(\hat f+\hat g\)的本征值就是二者本征值之和。

              对于没有共同本征态的两个物理量,其算符之和在任意态中的平均值为两者均值之和。
            \end{law}
            \begin{theorem}
              若\(f_0\)和\(g_0\)分别为\(\hat f\)和\(\hat g\)的最小本征值,则
              \begin{equation}
                (f+g)_0\geqslant f_0+g_0
              \end{equation}
              等号当且仅当\(f\)和\(g\)可以同时测量时成立。
            \end{theorem}
            




            \chapter{有心力场}
            \section{库仑力场}
            \begin{attr}
                \hl{库仑简并}:库仑力场的性质:每一个本征值对\(l\)是简并的,第\(n\)个能级的简并度为
                \begin{equation}
                    \sum_{l=0}^{n-1}(2l+1)=n^2
                \end{equation}
            \end{attr}
            \begin{theorem}
                算符\(\hat A=\frac{\vec r} r-\frac12(\vec p\times\hat l-\hat l\times \vec p)\)是守恒量
            \end{theorem}
            \begin{commute}
                \begin{align}
                    \akuo{\hat l_i,\hat A_j}&=i\epsilon_{ijk}\hat A_k\\
                    \com{\hat A_i}{\hat A_j}&=-2i\hat H\ep{ijk}\hat l_k
                \end{align}
            \end{commute} 

            \section{抛物坐标下库仑场中的运动}
            \begin{intro}
                抛物坐标下的拉普拉斯算子:
                \begin{equation}
                    \Delta=\frac4{\xi+\eta}\akuo{\pian{}{\xi}\xkuo{\xi\pian{}{\xi}}+\pian{}{\eta}\xkuo{\eta\pian{}{\eta}}}+\frac{1}{\xi\eta}\ppian{}{\varphi}
                \end{equation}
            \end{intro}
            \begin{deduce}
                库仑力场
                \begin{equation}
                    U=-\frac1r=-\frac{2}{\xi+\eta}
                \end{equation}
                中的单粒子薛定谔方程为:
                \begin{align}
                    0&=\frac12\Delta \psi+\hat H\psi\\
                    \label{eqn:pkl}&=\frac2{\xi+\eta}\akuo{\pian{}{\xi}\xkuo{\xi\pian{\psi}{\xi}}+\pian{}{\eta}\xkuo{\eta\pian{\psi}{\eta}}}+\frac{1}{2\xi\eta}\ppian{\psi}{\varphi}+\xkuo{E+\frac{2}{\xi+\eta}}\psi
                \end{align}
            \end{deduce}
            \begin{deduce}
                
                对\autoref{eqn:pkl}利用分离变量法可得
                \begin{equation}
                    \label{eqn:pklfl}
                    \begin{cases}
                        f_\varphi(\varphi)=e^{im=\varphi}\\
                        \dao{}{\xi}\xkuo{\xi\dao{f_\xi}{\xi}}+\akuo{\frac12E\xi-\frac{m^2}{4\xi}+\beta_\xi}f_\xi=0\\
                        \dao{}{\eta}\xkuo{\eta\dao{f_\eta}{\eta}}+\akuo{\frac12E\eta-\frac{m^2}{4\eta}+\beta_\eta}f_\eta=0
                    \end{cases}
                \end{equation}
                其中\(m\in\mathbb Z,\,\beta_\xi+\beta_\eta=1\)
            \end{deduce}
            \begin{theorem}
                由\autoref{eqn:pklfl}可得离散谱的定态由三个整数确定,分别称为\hl{抛物量子数}\(n_1,n_2\)和\hl{磁量子数}\(m\),其\hl{主量子数的表达式}为
                \begin{equation}
                    n=n_1+n_2+|m|+1
                \end{equation}
            \end{theorem}
            \begin{deduce}
                归一化的波函数为:
                \begin{equation}
                    \psi_{n_1n_2m}=\frac{\sqrt[]{2}}{n^2}f_{n_1m}\xkuo{\frac{\xi}{n}}f_{n_2m}\xkuo{\frac{\eta}{n}}\frac{e^{im\varphi}}{\sqrt[]{2\pi}}
                \end{equation}
                其中
                \begin{equation}
                    f_{pm}(\rho)=\frac{1}{|m|!}\sqrt[]{\frac{(p+|m|)!}{p!}}F(-p,|m|+1,\rho)e^{-\frac12\rho}\rho^{\frac12|m|}
                \end{equation}
            \end{deduce}
            \chapter{微扰论}
            \begin{method}
              由于薛定谔方程很难得到精确解,但忽略一些小量即可得到很好的解,于是可以通过\hl{微扰论}的方法把问题分为两步:
              \begin{enumerate}
                \item 求出忽略小量的简化问题的精确解
                \item 计算忽略小量引起的误差
              \end{enumerate}
            \end{method}
            \section{与时间无关的微扰}
            \begin{assmp}
              设哈密顿量呈以下形式:
              \begin{equation}
                \hat H=\hat H_0+\hat V
              \end{equation}
              其中算符\(\hat V\)代表微扰项,而\(\hat H_0\)代表未受扰项,其解已知:
              \begin{equation}
                \hat H_0\psi^{(0)}=E^{(0)}\psi^{(0)}
              \end{equation}
              假设\(\hat H_0\)无简并。
            \end{assmp}
            \begin{theorem}
              设\(\psi=\sum\limits_mc_m\psi_m^{(0)}\)可得
              \begin{equation}\label{eqn:disturb}
                \xkuo{E-E_k^{(0)}}c_k=\sum_mV_{km}c_m
              \end{equation}
              将第\(n\)个本征态的本征值的一级近似
              \begin{equation}
                E_n=E_n^{(0)}+E_n^{(1)}
              \end{equation}
              代入\autoref{eqn:disturb}可得
              \begin{equation}
                E_n^{(1)}=V_{nn}=\int\psi_n^{(0)*}\hat V\psi_n^{(0)}dq
              \end{equation}
              同时,其本征函数的一级近似
              \begin{equation}
                \psi_n=\psi_n^{(0)}+\psi_n^{(1)}
              \end{equation}
              代入可得
              \begin{equation}
                \psi_n^{(1)}=\sum_{m\neq n}\frac{V_{mn}}{E_n^{(0)}-E_m^{(0)}}\psi_m^{(0)}
              \end{equation}
              因此,要使得微扰项足够小(一阶小量)需要满足的条件为:
              \begin{equation}
                |V_{mn}|<<|E_n^{(0)}-E_m^{(0)}|
              \end{equation}
              进而可以算出物理量\(f\)的微扰:
              \begin{equation}
                f_{nm}=f_{nm}^{(0)}+\sum_{k\neq n}\frac{V_{nk}f_{km}^{(0)}}{E_n^{(0)}-E_k^{(0)}}+\sum_{k\neq m}\frac{V_{km}f_{nk}^{(0)}}{E_m^{(0)}-E_k^{(0)}}
              \end{equation}
              更高阶的近似同理
            \end{theorem}

            \section{久期方程}
            \begin{assmp}
              假设\(\hat H_0\)具有简并,而\(\psi_{n'},n'\in\Lambda \)代表能级\(E_n^{(0)}\)下的一套本征函数。
            \end{assmp}
            \begin{theorem}
              通过同样的方法可以得到
              \begin{equation}
                \sum_{n,n'\in \Lambda }\xkuo{V_{nn'}-E^{(1)}\delta_{nn'}}c_{n'}^{(0)}=0
              \end{equation}
              若要此方程有非零解,则需要
              \begin{equation}\label{eqn:seculareqn}
                |V_{nn'}-E^{(1)}\delta_{nn'}|=0
              \end{equation}
              \autoref{eqn:seculareqn}称为\hl{久期方程}。
            \end{theorem}
            \section{与时间有关的微扰}
            \begin{assmp}
              若微扰项含时间,而未扰系统的定态波函数为\(\Psi_k^{(0)}\),设受扰方程的解为
              \begin{equation}\label{eqn:timedistsol}
                \Psi=\sum_ka_k(t)\Psi_k^{(0)}
              \end{equation}
            \end{assmp}

            \begin{theorem}
              将\autoref{eqn:timedistsol}代入薛定谔方程后,可求得
              \begin{equation}
                i\hslash\dao{a_m}{t}=\sum_kV_{mk}(t)a_k
              \end{equation}
              其中
              \begin{equation}
                V_{mk}(t)=\int\Psi_m^{(0)*}\hat V\Psi_k^{(0)}dq=V_{mk}e^{i\omega_{mk}t}
              \end{equation}
              \begin{equation}
                \omega_{mk}=\frac{E_m^{(0)}-E_k^{(0)}}{\hslash}
              \end{equation}
              当未受扰
            \end{theorem}

            \section{有限时间作用下的跃迁}
            \begin{assmp}
              假定微扰\(V(t)\)作用于有限时间内(或者当\(t\rightarrow\pm \infty\)时,\(V(t)\)足够快地趋于零)。
            \end{assmp}
            \begin{assmp}
              假定微扰作用前,系统处于离散谱的第n个态中,随后由\(\Psi=\sum_ka_{kn}\Psi_k^{(0)}\)确定。
            \end{assmp}
            \section{周期微扰作用下的跃迁}
            朗道P140
            \section{连续谱中的跃迁}
            \begin{assmp}
              在初始时刻,系统处于某一为受扰的态,而后收到恒定微扰,求跃迁到具有同一能量的另一态的概率。
            \end{assmp}
            


        
        
        \end{document}