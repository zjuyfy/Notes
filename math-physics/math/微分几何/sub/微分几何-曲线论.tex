\documentclass[12pt, a4paper, oneside]{ctexbook}
        \usepackage{amsmath, amsthm, amssymb, amsfonts, bm, graphicx, hyperref, mathrsfs}
        \usepackage{tcolorbox}
        \usepackage{booktabs,array}
        \usepackage{tikz, xcolor, environ, xparse, zhnumber}
        
        %设置页眉
        \usepackage{fancyhdr}
        \pagestyle{fancy}
        
        
        
        
        
        
        \usetikzlibrary{shapes, decorations}
        
        %定义颜色
        \definecolor{dedcol}{RGB}{150,150,30}%推论环境的主色
        \definecolor{theocol}{RGB}{40,150,30}%定力环境的主色
        \definecolor{thrmcol}{RGB}{18,29,80}%默认定理等环境的背景色
        \definecolor{thrmedge}{RGB}{12,133,211}%默认定理等环境的边界颜色
        \definecolor{hyperlinkcol}{RGB}{32,112,102}%链接颜色
        \definecolor{hyperfilecol}{RGB}{135,206,235}%文件颜色
        \definecolor{hyperurlcol}{RGB}{3,168,158}%网址颜色
        \definecolor{hypercitecol}{RGB}{150,140,130}%引用颜色
        \definecolor{hlback}{RGB}{207,255,207}%高亮颜色
        \definecolor{opcol}{RGB}{235,125,75}%op颜色
        \definecolor{facecol}{RGB}{122,180,245}%封面颜色
        \definecolor{pscol}{RGB}{44,80,99}%图案颜色
        
        %定义hyper的颜色
        \hypersetup{
          colorlinks=true,
          linkcolor=hyperlinkcol,
          filecolor=hyperfilecol,
          urlcolor=hyperurlcol,
          citecolor=hypercitecol,
        }
        
        %定义字体
        \setCJKfamilyfont{hwxk}{华文行楷}
        \newcommand{\huawenxingkai}{\CJKfamily{hwxk}}
        \setCJKfamilyfont{hwkt}{华文楷体}
        \newcommand{\huawenkaiti}{\CJKfamily{hwkt}}
        \setCJKfamilyfont{hwhp}{华文琥珀}
        \newcommand{\huawenhupo}{\CJKfamily{hwhp}}
        \setCJKfamilyfont{hwls}{华文隶书}
        \newcommand{\huawenlishu}{\CJKfamily{hwls}}
        \setmainfont{华文楷体}
        
        %定义高亮
        \newtcbox{\hlbox}[1][red]{on line, arc = 2pt, outer arc = 0pt,
          colback = hlback, colframe = #1!50!black,
          boxsep = 0pt, left = 1pt, right = 1pt, top = 2pt, bottom = 2pt,
          boxrule = 0pt, bottomrule = 1pt, toprule = 1pt}
        \newcommand{\hl}[1]{\hlbox{#1}}
        \newcommand{\optxt}[1]{\textcolor{opcol}{#1}}
        
        
        
        
        
        %定义公式环境
        \newcommand{\newfancytheoremstyle}[5]{%
          \tikzset{#1/.style={draw=#3, fill=#2,very thick,rectangle,
              rounded corners, inner sep=10pt, inner ysep=20pt}}
          \tikzset{#1title/.style={fill=#3, text=#2}}
          \expandafter\def\csname #1headstyle\endcsname{#4}
          \expandafter\def\csname #1bodystyle\endcsname{#5}
        }
        
        \newfancytheoremstyle{fancythrm}{thrmcol!5}{thrmedge}{\huawenhupo}{\huawenxingkai}
        
        \makeatletter
        \DeclareDocumentCommand{\newfancytheorem}{ O{\@empty} m m m O{fancythrm} }{
          %% 定义计数器
          \ifx#1\@empty
            \newcounter{#2}
          \else
            \newcounter{#2}[#1]
            \numberwithin{#2}{#1}
          \fi
          %% 定义 "newthem" 环境
          \NewEnviron{#2}[1][{}]{%
            \noindent\centering
            \begin{tikzpicture}
              \node[#5] (box){
                \begin{minipage}{0.93\columnwidth}
                  \csname #5bodystyle\endcsname \BODY~##1
                \end{minipage}};
              \node[#5title, right=10pt] at (box.north west){
                {\csname #5headstyle\endcsname #3 \stepcounter{#2}\csname the#2\endcsname\; ##1}};
              \node[#5title, rounded corners] at (box.east) {#4};
            \end{tikzpicture}
          }[\par\vspace{.5\baselineskip}]
        }
        
        
        \makeatother
        
         % 定义各个环境的的样式
         % \newfancytheoremstyle{<name>}{inner color}{outer color}{head style}{body style}
        \newfancytheoremstyle{fancytheo}{theocol!5}{theocol}{\huawenhupo}{\huawenxingkai}
        \newfancytheoremstyle{fancyded}{dedcol!5}{dedcol}{\huawenhupo}{\huawenxingkai}
        
         % 定义各个新环境
         % \newfancytheorem[<number within>]{<name>}{<head>}{<symbol>}[<style>]
        \newfancytheorem[chapter]{define}{定义}{$\clubsuit$}
        \newfancytheorem[chapter]{claim}{声明}{$\clubsuit$}
        \newfancytheorem[section]{deduce}{推论}{$\heartsuit$}[fancyded]
        \newfancytheorem[section]{theorem}{定理}{$\spadesuit$}[fancytheo]
        
        \title{{\Huge{标题}}}
        \author{wave}
        \date{\today}
        \linespread{1.5}
        
        %设置章节标题样式\usepackage[english]{babel}
        \usepackage{blindtext}
        
        \usepackage[sc,compact,explicit]{titlesec} % Titlesec for configuring the header
        
        
        \usepackage{auto-pst-pdf} % Vectorian 装饰图案的 XeTeX 辅助 (见: https://tex.stackexchange.com/questions/253477/how-to-use-psvectorian-with-pdflatex)
        \usepackage{psvectorian} % Vectorian 中的装饰图案
        
        \let\clipbox\relax % PSTricks 已经定义了 \clipbox, 所以要去掉
        \usepackage{adjustbox} % 调整图案大小的
        
        \newcommand{\otherfancydraw}{% 定义图案
        \begin{adjustbox}{max height=0.5\baselineskip}% 根据行距设定高度,自己定
          \raisebox{-0.25\baselineskip}{
          \rotatebox[origin=c]{0}{% 旋转,自己定
            \psvectorian{84}% 图案,编号见 (http://melusine.eu.org/syracuse/pstricks/vectorian/psvectorian.pdf)
          }}%
        \end{adjustbox}%
        }
        
        % 画一条中间为图案的线 (见: https://tex.stackexchange.com/questions/15119/draw-horizontal-line-left-and-right-of-some-text-a-single-line/15122#15122)
        \newcommand*\ruleline[1]{\par\noindent\raisebox{.8ex}{\makebox[\linewidth]{\hrulefill\hspace{1ex}\raisebox{-.8ex}{#1}\hspace{1ex}\hrulefill}}}
        
        \titleformat% Formatting the header
          {\chapter} % command
          [block] % shape - Only managed to get it working with block
          {\normalfont\huawenlishu\huge} % format - Change here as needed
          {\centering 第\zhnum{chapter}章\\ \vspace{-0.6em}} % The Chapter N° label
          {0pt} % sep
          {\centering \ruleline{\otherfancydraw}\\ \vspace{-0.6em} % The horizontal rule
          \centering #1} % And the actual title
        
          \titleformat{\section}[block]{\huawenlishu\Large}{\thesection}{0pt}{\centering #1}

        %更改autoref的形式
        \def\equationautorefname{式}
        \def\footnoteautorefname{脚注}
        \def\itemautorefname{项}
        \def\figureautorefname{图}
        \def\tableautorefname{表}
        \def\appendixautorefname{附录}
        \def\chapterautorefname{章}
        \def\sectionautorefname{小节}
        \def\theoremautorefname{定理}


        \newcommand{\xkuo}[1]{\left(#1\right)}
        \newcommand{\dkuo}[1]{\left\lbrace#1\right\rbrace}
        \newcommand{\akuo}[1]{\left[#1\right]}
        \newcommand{\jkuo}[1]{\left\langle#1\right\rangle}


        %微积分
        \newcommand{\pian}[2]{\frac{\partial #1}{\partial #2}}
        \newcommand{\ppian}[2]{\frac{\partial^2 #1}{\partial #2^2}}
        \newcommand{\dao}[2]{\frac{d#1}{d#2}}
        \newcommand{\ddao}[2]{\frac{d^2#1}{d#2^2}}
        \newcommand{\ji}[2]{\int_{#1}^{#2}}
        \newcommand{\jji}[1]{\iint\limits_{#1}}

        
        

        
        \begin{document}
          \renewcommand*{\psvectorianDefaultColor}{pscol}%设定图案颜色
        
        
        
            \newpage
            \pagenumbering{alph}
            \setcounter{page}{1}
            \tableofcontents
            \newpage
            \setcounter{page}{1}
            \pagenumbering{arabic}
        
    
            \chapter{曲线论}
            \section{正则参数曲线}
            \begin{claim}
                本章节用\(E^3\)表示三维欧式空间
            \end{claim}
        
            \begin{define}
                \(E^3\)中的一条\hl{曲线C}定义为从区间\([a,b]\)到\(E^3\)的连续映射,称为参数曲线。

                若在\(E^3\)中取定一个正交标架\(\dkuo{O;\vec i,\vec j,\vec k}\),则曲线C上任意点\(p(t),t\in[a,b]\)相当于向量\(\overrightarrow{Op(t)}\),用\(\vec r(t)\)表示之
            \end{define}

            \begin{define}
                若满足以下两个条件
                \begin{enumerate}
                    \item \(\vec r(t)\)至少是\(t\)的三次以上连续可微的向量函数
                    \item \(\vec r'(t)\neq 0\)
                \end{enumerate},则称此曲线为\hl{正则参数曲线}

                而以t增大的方向为\hl{正方向}
            \end{define}


            \begin{deduce}
                正则曲线之间参数变换满足:
                \begin{enumerate}
                    \item \(t(u)\)是u的三次以上连续可微函数
                    \item \(t'(u)\)处处不为零
                \end{enumerate}

                参数变换在\(r(t)\)和\(r(u)\)之间建立了等价关系。由全体等价的正则参数曲线构成的集合称为一条\hl{正则曲线}。

                若\(t'(u)>0\),则此映射保持曲线的定向不变。
            \end{deduce}

            \begin{deduce}
                曲线还可以用隐函数表示:
                \begin{equation}\label{eq:implicitfun}
                    \begin{cases}
                        f(x,y,z)=0\\
                        g(x,y,z)=0
                    \end{cases}
                \end{equation}
                当矩阵
                \begin{equation}
                    \begin{pmatrix}
                        \pian fx&\pian fy &\pian fz\\
                        \pian gx &\pian gy &\pian gz
                    \end{pmatrix}
                \end{equation}
                的秩为2时,由隐函数定理,可以从方程组\autoref{eq:implicitfun}解出其中两个坐标作为另一个坐标的函数,以此得到参数方程。
            \end{deduce}

            \section{曲线的弧长}
            \begin{define}
                设\(\vec r=\vec r(t)\)为\(E^3\)中的一条正则曲线,则
                \begin{equation}\label{eq:arclength}
                    s=\ji ab|\vec r'(t)|dt
                \end{equation}
                是该曲线的一个不随直角坐标系选择而变化的量(因为\(|\vec r'|\)不变)。
                
                同时也和保持定向的参数无关:设\(t=t(u),\,t'(u)>0\),则有
                \begin{equation*}
                    \ji ab\left|\dao{\vec r(t)}{t}\right| dt=\ji{u(a)}{u(b)}\left|\dao{\vec r(t(u))}{t}\right|\cdot \dao tu du =\ji{u(a)}{u(b)}\left|\dao{\vec r(t(u))}{u}\right| du
                \end{equation*}
                不变量s的几何意义是该曲线\hl{弧长}
            \end{define}
            \begin{define}
                曲线可以用弧长\(s\)作为参数(参数变换由\autoref{eq:arclength}固定a且b换成t得到,是t的三次以上连续可微函数),称为\hl{弧长参数}
            \end{define}

            \begin{theorem}
                设\(\vec r=\vec r(t)\)是\(E^3\)中一条正则曲线,则t是它的弧长参数的充分必要条件为\(|\vec r'(t)|=1\)
            \end{theorem}

            \section{曲线的曲率和Frenet标架}
            \begin{claim}
                命\(\vec\alpha(s)=\vec r'(s)\),其中s为弧长参数。
            \end{claim}
            \begin{theorem}
                \begin{equation}
                    \lim_{\Delta s\rightarrow0}\left|\frac{\Delta\theta}{\Delta s}\right|=\left|\frac{d\vec \alpha}{ds}\right|
                \end{equation}
                其中\(\Delta \theta\)表示切向量\(\vec \alpha(s+\Delta s)\)与\(\vec \alpha (s)\)之间的夹角。
            \end{theorem}

            \begin{define}
                设曲线方程为\(\vec r(s)\),其中\(s\)是曲线的弧长参数,定义
                \begin{equation}
                    \kappa(s)=\left|\frac{d\vec \alpha}{ds}\right|=|\vec r''(s)|
                \end{equation}
                为曲线在\(s\)处的\hl{曲率},并且称\(\dao{\vec \alpha}{s}\)为该曲线的曲率向量。
            \end{define}
            \begin{theorem}
                曲线是一条直线当且仅当它的曲率\(\kappa(s)\equiv0\)
            \end{theorem}
            \begin{define}
                若\(\kappa(s)\neq0\),则向量\(\vec \alpha'(s)\)有确定的方向,此方向单位向量定义为曲线的\hl{主法向量}\(\vec \beta\):
                \begin{equation}
                    \vec\alpha'(s)=\kappa(s)\vec \beta(s)
                \end{equation}
                单位切向量\(\vec \alpha(s)\)和单位主法向量\(\vec \beta(s)\)唯一地确定了最后一个单位向量\hl{次法向量}\(\vec \gamma\):\begin{equation}
                    \vec \gamma(s)=\vec \alpha(s)\times\vec\beta(s)
                \end{equation}
                这三个单位向量构成\hl{Frenet标架}

                以\(\vec\alpha\)为法向量的平面称为\hl{法平面},以\(\vec \beta\)为法向量的平面称为\hl{切平面},以\(\vec \gamma\)为法向量的平面称为\hl{密切平面}
            \end{define}

            \section{曲线的挠率和Frenet公式}

            \begin{define}
                设\(\vec\beta(s)\)和\(\vec\gamma(s)\)分别是曲线的主法向量和次法向量,其中s为弧长参数,则定义\(\tau(s)=-\vec\gamma'(s)\cdot\vec\beta(s)\)为曲线的\hl{挠率}
            \end{define}


            \begin{theorem}
                曲线是平面曲线当且仅当它的挠率为0
            \end{theorem}

            \begin{deduce}
                \hl{Frenet公式:}
                \label{fomular:Frenet}
                \begin{equation}
                    \left\{ 
                    \begin{aligned}
                        \vec r'(s)= &\vec\alpha(s),\\
                        \vec \alpha'(s)= &&\kappa(s)\vec\beta(s),\\
                        \vec \beta '(s)= &-\kappa(s)&&+\tau(s)\vec\gamma(s),\\
                        \vec\gamma'(s)= &&-\tau(s)\vec\beta(s).
                    \end{aligned}\right.
                \end{equation}
                或者写成矩阵形式:
                \begin{equation}
                    \begin{pmatrix}
                        \vec \alpha'(s)\\
                        \vec \beta'(s)\\
                        \vec \gamma'(s)
                    \end{pmatrix}=\begin{pmatrix}
                        0&\kappa(s)&0\\
                        -\kappa(s)&0&\tau(s)\\
                        0&-\tau(s)&0
                    \end{pmatrix}\begin{pmatrix}
                        \vec \alpha(s)\\
                        \vec\beta(s)\\
                        \vec\gamma(s)
                    \end{pmatrix}
                \end{equation}
            \end{deduce}

            \begin{theorem}
                利用\href{fomular:Frenet}{Frenet公式}可以证明,若曲线的曲率\(\kappa(s)\)和挠率\(\tau(s)\)都不为零,s是弧长参数,那么该曲线落在一个球面上仅当他的曲率和挠率满足
                \begin{equation}
                    \xkuo{\frac{1}{\kappa(s)}}^2+\xkuo{\frac1{\tau(s)}\dao{}{s}\xkuo{\frac{1}{\kappa(s)}}}^2=\text{常数}
                \end{equation}
            \end{theorem}


            \begin{deduce}
                若t不是曲线的弧长参数,则
            \end{deduce}

            \begin{theorem}
                
            \end{theorem}

            \section{曲线论基本定理}




            
        
        \end{document}