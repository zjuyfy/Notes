
\documentclass{report}

\usepackage[fntef]{ctex}
\usepackage{amsmath}
\usepackage{amssymb}
\usepackage{amsthm}
\usepackage{mathrsfs}
\usepackage{color}
\usepackage{graphicx}
\usepackage{enumerate}
\usepackage{enumitem}
\usepackage{breqn}
\AddEnumerateCounter{\chinese}{\chinese}{}


\setCJKfamilyfont{hwxk}{华光行楷_CNKI}
\newcommand{\huawenxingkai}{\CJKfamily{hwxk}}
\setmainfont{HGB1X_CNKI}




\newcommand{\xkuo}[1]{\left(#1\right)}
\newcommand{\dkuo}[1]{\left\lbrace#1\right\rbrace}
\newcommand{\akuo}[1]{\left[#1\right]}
\newcommand{\jkuo}[1]{\left\langle#1\right\rangle}
\newcommand{\daa}[1]{\par\textcolor{blue}{\huawenxingkai #1}}
\newcommand{\wen}[1]{\mbox{#1}}
\newcommand{\you}{\mbox{又}}
\newcommand{\dang}{\mbox{当}}
\newcommand{\yyou}{\mbox{有}}
\newcommand{\qie}{\mbox{且}}
\newcommand{\jishu}[2]{\sum #1_n(#2)}
\newcommand{\jjishu}[2][x]{\sum |#2_n(#1)|}
\newcommand{\xti}[3]{\[(#1)#2\]\da{#3}}
\newcommand{\pian}[2]{\frac{\partial #1}{\partial #2}}
\newcommand{\ppian}[2]{\frac{\partial^2 #1}{\partial #2^2}}
\newcommand{\dao}[2]{\frac{d#1}{d#2}}
\newcommand{\ddao}[2]{\frac{d^2#1}{d#2^2}}
\newcommand{\cen}{^\circ C}
\newcommand{\fah}{^\circ F}
\newcommand{\ji}[2]{\int_{#1}^{#2}}
\newcommand{\qh}[1]{\sum\limits_{#1}}
\newcommand{\jji}[1]{\iint\limits_{#1}}
\newcommand{\ppi}{\frac\pi2}
\newcommand{\ege}{\frac{\sqrt2}{2}}
\newcommand{\e}[1]{\times10^{#1}}
\newcommand{\ti}[1]{\textbf{#1}}


\newcommand{\ldotfill}[2]{\leavevmode\xleaders\hbox{\rule{2pt}{0.4pt}\ }\hfill\null}

%作业用
\newcounter{que}
\setcounter{que}{1}
\newenvironment{question}[1][\theque]{\vspace*{2cm}\par\noindent\hrule\vspace*{2pt}\hrule\vspace*{10pt}\noindent\bfseries\large Ex#1.}{\stepcounter{que}}
\definecolor{anscolor}{RGB}{50,120,170}
\newenvironment{answer}[1][答]{\par\centerline{\makebox[10cm]{\dotfill}}\par\hangafter1\hangindent2em\noindent\textbf{#1.}\huawenxingkai\color{anscolor}\\}{\par}

%数学分析
\newcommand{\ya}[4]{\frac{\partial(#1,#2)}{\partial(#3,#4)}}
\newcommand{\zkya}[4]{\pd #1#3#4\pd #2#4#3-\pd #1#4#3\pd #2#3#4}

%复变函数
\newcommand{\wqji}{\int_{-\infty}^{\infty}}

%热学
\newcommand{\pd}[3]{\xkuo{\frac{\partial#1}{\partial#2}}_#3}

%原子物理
\newcommand{\bra}[1]{\left\langle #1 \right|}
\newcommand{\ket}[1]{\left| #1 \right\rangle}
\newcommand{\p}[1]{\partial_{#1}}%对下标的偏导
\newcommand{\ep}[1]{\epsilon_{#1}}%全反对称张量
\newcommand{\dt}[1]{\delta_{#1}}%delta张量

%理论力学
\newcommand{\keq}[2]{\pian{\mathscr{#1}}{#2}}
\newcommand{\zkps}[3]{\pian{#1}{q_#3}\pian{#2}{p_#3}-\pian{#1}{p_#3}\pian{#2}{q_#3}}

%正文
\newcommand{\sub}[1]{\(_{#1}\)}
\newcommand{\sps}[1]{\(^{#1}\)}


\begin{document}
    \begin{enumerate}[label = \chinese*、]
        \item 矢量运算
        \begin{enumerate}[label = (\arabic*)]
            \item \(\boldsymbol{A}\cdot(\boldsymbol{B}\times\boldsymbol{C})=\boldsymbol{B}\cdot(\boldsymbol{C}\times\boldsymbol{A})=\boldsymbol{C}\cdot(\boldsymbol{A})\)
            \item \(\boldsymbol{A}\times(\boldsymbol{B}\times\boldsymbol{C})=(\boldsymbol{A}\cdot\boldsymbol{C})\boldsymbol{B}-(\boldsymbol{A}\cdot\boldsymbol{B})\boldsymbol{C}\)
        \end{enumerate}
        \item 矢量微分
        \begin{enumerate}[label = (\arabic*)]
            \item \(\boldsymbol{\nabla}(fg)=f\boldsymbol{\nabla}g+g\boldsymbol{\nabla}f\)
            \item \(\boldsymbol{\nabla}(\boldsymbol{A}\cdot\boldsymbol{B})=\boldsymbol{A}\times(\boldsymbol{\nabla}\times\boldsymbol{B})+\boldsymbol{B}\times(\boldsymbol{\nabla}\times\boldsymbol{A})+(\boldsymbol{A}\cdot\boldsymbol{\nabla})\boldsymbol{B}+(\boldsymbol{B}\cdot\boldsymbol{\nabla})\boldsymbol{A}\)
            \item \(\boldsymbol{\nabla}\cdot(f\boldsymbol{A})=f(\boldsymbol{\nabla}\cdot \boldsymbol{A})+\boldsymbol{A}\cdot(\boldsymbol{\nabla}f)\)
            \item \(\boldsymbol{\nabla}\cdot(\boldsymbol{A}\times\boldsymbol{B})=\boldsymbol{B}\cdot(\boldsymbol{\nabla}\times\boldsymbol{A})-\boldsymbol{A}\cdot(\boldsymbol{\nabla}\times\boldsymbol{B})\)
            \item \(\boldsymbol{\nabla}\times(f\boldsymbol{A})=f(\boldsymbol{\nabla}\times\boldsymbol{A})-\boldsymbol{A}\times(\boldsymbol{\nabla}f)\)
            \item \(\boldsymbol{\nabla}\times(\boldsymbol{A}\times\boldsymbol{B})=(\boldsymbol{B}\cdot\boldsymbol{\nabla})\boldsymbol{A}-(\boldsymbol{A}\cdot\boldsymbol{\nabla})\boldsymbol{B}+\boldsymbol{A}(\boldsymbol{\nabla}\cdot\boldsymbol{B})-\boldsymbol{B}(\boldsymbol{\nabla}\cdot\boldsymbol{A})\)
        \end{enumerate}
        \item 矢量二阶微分
        \begin{enumerate}
            \item \(\boldsymbol{\nabla}\cdot(\boldsymbol{\nabla}\times\boldsymbol{A})=0\)
            \item \(\boldsymbol{\nabla}\times(\boldsymbol{\nabla}f)=0\)
            \item \(\boldsymbol{\nabla}\times(\boldsymbol{\nabla}\times\boldsymbol{A})=\boldsymbol{\nabla}(\boldsymbol{\nabla}\cdot\boldsymbol{A})-\boldsymbol{\nabla}^2\boldsymbol{A}\)
        \end{enumerate}
    \end{enumerate}
\end{document}