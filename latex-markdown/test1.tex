%导言区
\documentclass[10pt]{ctexart}%还有book,report,letter类;设置默认字体大小为10pt
%ctrl+T快速注释,ctrl+U快速取消注释
\bibliographystyle{plain}

\usepackage{ctex}%查看宏包帮助
\usepackage{xltxtra}%针对XeTeX的改进
\usepackage{texnames}%一些logo
\usepackage{mflogo}
\usepackage{graphicx}%图片
\usepackage{booktabs}%表格增加功能
\usepackage{longtable}%长表格功能
\usepackage{tabu}%综合表格功能
\usepackage{caption}%浮动体标题
\usepackage{bicaption}%同上
%\usepackage{subcaption}%同上
\usepackage{subfig}%子浮动体
\usepackage{floatrow}%
\usepackage{picinpar}%绕排
\usepackage{wrapfig}%同上
\usepackage{amsmath}
\usepackage{amssymb}
\graphicspath{{figure/},{pic/}}

\newcommand{\degree}{^\circ}%定义新的命令
\newcommand{\adots}{\mkern2mu\raisebox{0.1em}{.}\mkern2mu\raisebox{0.4em}{.}\mkern2mu\raisebox{0.7em}{.}\mkern1mu}%右斜点列
\newcommand{\shangbiao}[2]{$#1^{#2}$}
\newcommand{\biao}[3][^]{$#2#1#3$}
%\renewcommand{cmd}{def}重新定义已有命令
%\renewcommand{\abstractname}{内容简介}
\newenvironment{myabstract}[1][摘要]
{\small 
\begin{center}
	\bfseries #1
\end{center} 
\begin{quotation}}{\end{quotation}}



\title{test 1st\\\heiti 测试一}
\author{wave\\\kaishu 秦始皇}
\date{\today}%设置编辑时间

\ctexset{
}


%正文区(文稿区)
\begin{document}
	\maketitle
	\tableofcontents%目录
	\section{函数、环境测试区}
	\shangbiao{3}{ae}
	\biao{2}{3}
	\biao[_]{2}{3}
	\begin{myabstract}[abb]
		内容...
	\end{myabstract}
    $\mathrm{S}$
%	\begin{abstract}
%		简介
%	\end{abstract}
a\[ab\]



	\section{引言}
	\subsection{字体}
	\subsubsection{字体族设置}
	
	罗马字体\textrm{Roman Family},
	还有无衬线字体\textsf{Sans Serif Family}和打印机字体\texttt{Typewriter Family}或者
	\rmfamily%后面均为罗马字体
	Roman
	\sffamily
	Sans Serif
	\ttfamily
	Typerwriter
	
	\subsubsection{字体系列设置}
	
	\rmfamily
	\textmd{中等Medium}
	\textbf{粗体Boldface}
	{\mdseries Medium}{\bfseries Boldface}%只在在括号内影响
	
	\subsubsection{字体形状设置}
	
	\textup{直立Up Right}
	\textit{斜体Italy}
	\textsl{伪斜体Slanded}
	\textsc{小型大写Small Caps}
	当然也有\upshape upright shape
	
	\subsubsection{中文字体}
	
	{\songti 宋体}\quad{\heiti 黑体}\quad{\fangsong 仿宋}\quad{\kaishu 楷书}%quad类似于空格
	
	\subsubsection{字体大小(相较于默认大小)}
	
	{\tiny H1H}{\scriptsize H2H}{\footnotesize H3H}{\small H4H}{\normalsize H5H}{\large H6H}{\Large H7H}{\LARGE H8H}{\huge H9H}{\Huge H0H}
	
	\subsubsection{中文字体字号}
	
	{\zihao{0}初号}{\zihao{-2}小二号}{\zihao{-6}小六号}{\zihao{8}八号}
	\\
	\\
	\subsection{符号}
	\subsubsection{空格}
	空格\\
	小的空\thinspace 格\\
	小的空\ 格\\
	小的空\, 格\\
	半个空\enspace 格\\
	一个空\quad 格\\
	两个空\qquad 格\\
	硬空~格\\
	指定宽度的空\kern 1pc 格\\%1pc=12pt=4.128mm
	指定宽度的空\kern 1mm 格\\	
	指定宽度的空\kern 1pt 格\\
	指定宽度的空\kern -1em 格\\
	指定宽度的空格\\
	指定宽度的空\hskip 1em 格\\
	指定宽度的空\hskip 1cm 格\\
	指定宽度的空\hspace{20pt}格\\
	指定占位的空\hphantom{xxx}格\\%根据括号内字符占据宽度
	指定占位的空\hphantom{空格xx}格\\
	弹性长度(撑满\hfill 空间)的空格\\
	\subsubsection{特殊符号}
	被调用符号\\
	\# \$ \% \{\} \{ abc\} \~{a} \~{} \_{} \_{2} \^{a} \^{} \textbackslash \&
	
	排版符号\\
	\S \P \dag \ddag \copyright \pounds 
	
	标志符号\\
	\TeX{} \LaTeX{} \LaTeXe
	
	引号\\
	` ' `` ''
	
	连字符\\
	- -- ---
	
	非英文字符\\
	\oe \OE \ae \AE \aa \AA \o \O \l \L \ss	!` ?`
	
	重音符号\\
	\`o \'o \^o \''o \~o \.o \=o \u{o} \v{a} \H{o} \r{a} \t{o} \b{a} \c{o} \d{o}
	
	
	\subsection{图片}
	灵梦:
	
	\includegraphics[scale=0.1]{lm}
	\includegraphics[height=2cm]{lm}\\
	\includegraphics[width=10cm]{lm}\\
	\includegraphics[height=0.5\textheight]{lm}\\%一页文本高度的一般
	\includegraphics[width=\linewidth]{lm}\\%一行宽度
	一\hfill 行\\
	\includegraphics[height=2em]{lm}%测试用
	\includegraphics[height=3cm,angle=52]{lm}
	
	\subsection{表格}
	\begin{tabular}{l||c c c| p{1.5cm}}%l左对齐,c居中,r右对齐,p指定列宽度
		姓名 & 出生 & 事业 & 成就 & 备注\\
		\hline
		under & 很早 & 物理 & 太空 & 不用吃饭\\
		physics & 较晚 & 人类 & 有序 & 喜欢\\
		\hline \hline
		结束 & 很晚 & 结束 & 虚无 & 无需一切东西\\
	\end{tabular}\\
    \subsection{浮动体}%管理图片表格
    图片浮动体\\
    
    灵梦见图\ref{lm}
    \begin{figure}[h]%h:here,b:bottom,t:top,p:page
    	\centering%居中
    	\includegraphics[height=3cm]{lm}
    	\caption{灵梦}
    	\label{lm}
    \end{figure}

    各类浮动体\\
    
    表格浮动体与图片基本一致:\\
    \begin{table}[h]
    \caption{梦想}
    \centering
    	    \begin{tabular}{l||c c c| p{1.5cm}}%l左对齐,c居中,r右对齐,p指定列宽度
    		姓名 & 出生 & 事业 & 成就 & 备注\\
    		\hline
    		under & 很早 & 物理 & 太空 & 不用吃饭\\
    		physics & 较晚 & 人类 & 有序 & 喜欢\\
    		\hline \hline
    		结束 & 很晚 & 结束 & 虚无 & 无需一切东西\\
    	\end{tabular}
    \end{table}




	\section{正文}
	Hello world!\\你好!
	
	Let $f(x)$ be defined by the formula
	$$f(x)={3x_{1}^{2}+x_{1}-1}/{x\sqrt{x}}$$%
	 which is a polynomial of degree 2.
	
	勾股定理:\par %分段
	设直角三角形 \(ABC\) ,其中$\angle C=90\degree $%行内公式
	,则有:
	\begin{equation}
		AB^2 = BC^2 + AC^2
	\end{equation}%产生带编号的公式
	
	当然也有余弦定理\begin{math}
		c^2=a^2+b^2-2abcos(C)
	\end{math}%行内
	\subsection{数学符号}
	乘号(叉乘):$a \times b$
	\subsection{上下标}
	上标:$x^2+x^{2^2}$
	
	下标:$a_0+a_1$
	
	\subsection{希腊字母}
	alpha$\alpha$, beta$\beta$,omega$\Omega$ , omega$\omega$......
	\subsection{函数}
	$y=\arcsin x arcsin x$
	
	$z=\log_2 x$
	
	$f=\sqrt{x^3+y^2}$
	
	$f(x)=\sqrt[n]{x^4}$
	\subsection{分式}
	$3/4$或$\frac{x^2t}{x_0}$
	
	$\frac{1}{1+\frac{1}{x}}$
	
	$\sqrt{1-\frac{v^2}{c^2}}$
	\subsection{行内行间公式}
	行$x\sin x$内
	
	行\(x\cos x\)内
	
	行间$$\frac{x}{\sin x}$$公式。
	
	行间\[\sqrt{1-\sin^2 x}\]公式。
	
	行间
	\begin{displaymath}
		x*\sin x*\cosh x.
		\text{\large \heiti 数学模式中临时切换到文本模式}\\
		\mbox{第二种插入文本方式}
		x+1=xx
	\end{displaymath}
	环境。
	
	行间自动编号
	\begin{equation}
		g_{ik;l}=-g^{ik}_{;l}
		\label{度规}
	\end{equation}
	环境(用快捷键ctrl+shift+N)

	行间不自动编号
	\begin{equation*}
		a+b=b+a
		\label{abel}
	\end{equation*}
	环境
	
	\subsection{矩阵}
	$
	\begin{matrix}
		1&2&3\\
		4&5&6\\
		7&8&9
	\end{matrix}
	\begin{pmatrix}
        1&2&3\\
        4&5&6
	\end{pmatrix}
	\begin{bmatrix}
	    1&2&3\\
	    4&x&y
	\end{bmatrix}
	\begin{Bmatrix}
	    a_{11}&a_{12}\\
	    a_{21}&a_{22}
	\end{Bmatrix}
	\begin{vmatrix}
        a&ab\\
        ba&b
	\end{vmatrix}
	\begin{Vmatrix}
	    aa&cc\\
	    bb&dd
	\end{Vmatrix}
	\begin{pmatrix}
	    a_{11}&\dots &a_{1n}\\
	    \vdots&\ddots&\vdots\\
	    a_{n1}&\dots &a_{nn}
	\end{pmatrix}
	$
	
	$
	\begin{pmatrix}
	    a_{11}&       &\multicolumn{2}{c}{\raisebox{-1ex}[0 pt]{\large 0}}\\
	    \hdotsfor{4}\\%跨列省略号
	          &a_{22} &         &       \\
	          &       &\ddots   &       \\
	   \multicolumn{2}{c}{\raisebox{1ex}[0 pt]{\large 0}}&    &a_{nn}       
	\end{pmatrix}%合并多列
	$
	
	以及行内小矩阵\(\left(\begin{smallmatrix}
	a&b\\
	c&d
	\end{smallmatrix}\right)\)
	
	\[\begin{array}{|r|r|}%格式与表格同
	\frac{a}{b} &0\\
	\hline
	a\frac{b}{c}&0
	\end{array}\]
	
	复杂矩阵:
	\[
	\begin{array}{c@{\hspace{-3mm}}l}%@内容不计数,只计空格
	   \left(
	   \begin{array}{ccc|ccc}
	      a&\cdots&a&b&\cdots&b\\
	      &\ddots&\vdots&\vdots&\adots&\\
	      &&a&b&&\\
	      \hline
	      &&&c&\cdots&c\\
	      \multicolumn{3}{c|}{\raisebox{0em}{\Huge 0}}&\vdots&&\vdots\\%raisebox向上抬
	      &&&c&\cdots&c
	   \end{array}
	   \right)
	   &
	   \begin{array}{l}
	   \left.\rule{0mm}{2em}\right\}p\\
	   \\
	   \left.\rule{0mm}{2em}\right\}p
	   \end{array}
	   \\[-5pt]%行距
	   \begin{array}{c@{\hspace{5mm}}c}
	   \underbrace{\rule{4em}{0mm}}_m&%underbrace下括号,rule尺码
	   \underbrace{\rule{4em}{0mm}}_m
	   \end{array}
	   &
	\end{array}
	\]
	$
	\raisebox{0.1em}{.}\raisebox{0.4em}{.}\raisebox{0.7em}{.}
	$
	\subsection{多行公式排版}
	\begin{gather}%gather*不带编号
		a+b=b+a\\
		3a+2b=7\\
		41ab+c^2=\mathcal{L} \notag%阻止编号
	\end{gather}
	
	公式对齐:
	\begin{align}%align*不带编号
		xx+a_{12}&=0\\
		y&=x^2+1
	\end{align}
	
	多行公式对应一个指标:
	\begin{equation}
		\begin{split}
		\sin^2x +\cos^2x &=1\\
		1+\tan^2x &=\sec^2x
		\end{split}
	\end{equation}
	
	分段函数:
	\begin{equation}
		D(x)=\begin{cases}
		1&x\in \mathbb{Q}\\
		0&\text{其它}
		\end{cases}
	\end{equation}
    \begin{equation}
    	\left(\begin{array}{c} - \mathrm{log}\!\left(\frac{\sqrt{4\, L^2\, g^2\, t^2 - 4\, L^2\, v^2 - g^2\, t^4\, v^2 + 4\, g\, h\, t^2\, v^2 - 4\, h^2\, v^2}\, \mathrm{i} + L\, g\, t\, 2\, \mathrm{i}}{v\, \left(\mathrm{i}\, g\, t^2 + 2\, L - h\, 2\, \mathrm{i}\right)}\right)\, \mathrm{i}\\ - \mathrm{log}\!\left(-\frac{\sqrt{4\, L^2\, g^2\, t^2 - 4\, L^2\, v^2 - g^2\, t^4\, v^2 + 4\, g\, h\, t^2\, v^2 - 4\, h^2\, v^2}\, \mathrm{i} - L\, g\, t\, 2\, \mathrm{i}}{v\, \left(\mathrm{i}\, g\, t^2 + 2\, L - h\, 2\, \mathrm{i}\right)}\right)\, \mathrm{i} \end{array}\right)
    \end{equation}
    \section{参考文献}
    引用了\cite{me}以及\cite{sci}还有\cite{lit}
    \begin{thebibliography}{99}
    	\bibitem{me}何逸阳.\emph{学习}[J].计算机.2020.11
    	\bibitem{sci}刘慈欣.\emph{三体\_死神永生}cc,bb,2020
	    \bibitem{lit}托尔斯泰.\emph{安娜$\bullet$卡列尼娜}.\texttt{www.baidu.com}
	    
    \end{thebibliography}
    \bibliography{landau}%编译参考文献文件
    \cite{mitt}%引用文献文件(文件可以从google学术搜索里点击“引用”或从zotero、从知网上获取)
\end{document}


